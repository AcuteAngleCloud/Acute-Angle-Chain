Chain\+Base is designed to meet the demanding requirments of blockchain applications, but is suitable for use in any application that requires a robust transactional database with the ability have near-\/infinate levels of undo history.

While chainbase was designed for blockchain applications, it is suitable for any program that needs to persist complex application state with the ability to undo.

\subsection*{Features}


\begin{DoxyItemize}
\item Supports multiple objects (tables) with multiple indicies (based upon boost\+::multi\+\_\+index\+\_\+container)
\item State is persistant and sharable among multiple processes
\item Nested Transactional Writes with ability to undo changes
\end{DoxyItemize}

\subsection*{Dependencies}


\begin{DoxyItemize}
\item c++11
\item \href{http://www.boost.org/}{\tt Boost}
\item C\+Make Build Process
\item Supports Linux, Mac OS \mbox{\hyperlink{class_x}{X}} (no Windows Support)
\end{DoxyItemize}

\subsection*{Example Usage}


\begin{DoxyCode}
enum tables \{
   book\_table
\};

/**
 * Defines a "table" for storing books. This table is assigned a 
 * globally unique ID (book\_table) and must inherit from chainbase::object<> which
 * decorates the book type by defining "id\_type" and "type\_id" 
 */
struct book : public chainbase::object<book\_table, book> \{

   /** defines a default constructor for types that don't have
     * members requiring dynamic memory allocation.
     */
   CHAINBASE\_DEFAULT\_CONSTRUCTOR( book )

   id\_type          id; ///< this manditory member is a primary key
   int pages        = 0;
   int publish\_date = 0;
\};

struct by\_id;
struct by\_pages;
struct by\_date;

/**
 * This is a relatively standard boost multi\_index\_container definition that has three 
 * requirements to be used withn a chainbase database:
 *   - it must use chainbase::allocator<T> 
 *   - the first index must be on the primary key (id) and must be unique (hashed or ordered)
 */
typedef multi\_index\_container<
  book,
  indexed\_by<
     ordered\_unique< tag<by\_id>, member<book,book::id\_type,&book::id> >, ///< required 
     ordered\_non\_unique< tag<by\_pages>, BOOST\_MULTI\_INDEX\_MEMBER(book,int,pages) >,
     ordered\_non\_unique< tag<by\_date>, BOOST\_MULTI\_INDEX\_MEMBER(book,int,publish\_date) >
  >,
  chainbase::allocator<book> ///< required for use with chainbase::database
> book\_index;

/**
    This simple program will open database\_dir and add two new books every time
    it is run and then print out all of the books in the database.
 */
int main( int argc, char** argv ) \{
   chainbase::database db;
   db.open( "database\_dir", database::read\_write, 1024*1024*8 ); /// open or create a database with 8MB
       capacity
   db.add\_index< book\_index >(); /// open or create the book\_index 


   const auto& book\_idx = db.get\_index<book\_index>().indicies();

   /**
      Returns a const reference to the book, this pointer will remain
      valid until the book is removed from the database.
    */
   const auto& new\_book300 = db.create<book>( [&]( book& b ) \{
       b.pages = 300+book\_idx.size();
   \} );
   const auto& new\_book400 = db.create<book>( [&]( book& b ) \{
       b.pages = 300+book\_idx.size();
   \} );

   /**
      You modify a book by passing in a lambda that receives a
      non-const reference to the book you wish to modify. 
   */
   db.modify( new\_book300, [&]( book& b ) \{
      b.pages++;
   \});

   for( const auto& b : book\_idx ) \{
      std::cout << b.pages << "\(\backslash\)n";
   \}

   auto itr = book\_idx.get<by\_pages>().lower\_bound( 100 );
   if( itr != book\_idx.get<by\_pages>().end() ) \{
      std::cout << itr->pages;
   \}

   db.remove( new\_book400 );

   return 0;
\}
\end{DoxyCode}


\subsection*{Concurrent Access}

By default Chain\+Base provides no synchronization and has the same concurrency restrictions as any boost\+::multi\+\_\+index\+\_\+container. This means that two or more threads may read the database at the same time, but all writes must be protected by a mutex. ~\newline
 Multiple processes may open the same database if care is taken to use interpocess locking on the database. ~\newline
 \subsection*{Persistance}

By default data is only flushed to disk upon request or when the program exits. So long as the program does not crash in the middle of a call to db.\+modify(), or db.\+create() the content of the database should remain in a consistant state. This means that you should minimize the complexity of the lambdas used to create and/or modify state.

If the operating system crashes or the computer loses power, then the database will be left in an undefined state depending upon which memory pages that operating system was able to sync to disk.

Chain\+Base was designed to be used with blockchain applications where an append-\/only log of blocks is used to secure state in the event of power loss. This block log can be replayed to regenerate the full database state. Dealing with OS crashes, loss of power, and logs, is beyond the scope of Chain\+Base.

\subsection*{Portability}

The contents of the database file is dependent upon the memory layout of the computer and process that created the database. Moving the database to a machine that uses a different compiler, operating system, libraries, or build type (release vs debug) will result in undefined behavior. ~\newline
 If portability is desired, the developer will have to export the database to a suitable format.

\subsection*{Background}

Blockchain applications depend upon a high performance database capable of millions of read/write operations per second. Additionally blockchains operate on the basis of \char`\"{}eventually consistant\char`\"{} which means that any changes made to the database are potentially reversible for an unknown amount of time depending upon the consenus protocol used.

Existing database such as \href{https://github.com/libbitcoin/libbitcoin-database}{\tt libbitcoin Database} achieve high peformance using similar techniques (memory mapped files), but they are heavily specialised and do not implement the logic necessary for multiple indicies or undo history.

Databases such as Level\+DB provide a simple Key/\+Value database, but suffer from poor performance relative to memory mapped file implementations. 