Web\+Socket++ is a C++ library that can be used to implement Web\+Socket functionality. The goals of the project are to provide a Web\+Socket implementation that is portable, flexible, lightweight, low level, and high performance.

Web\+Socket++ does not intend to be used alone as a web application framework or full featured web services platform. As such the components, examples, and performance tuning are geared towards operation as a Web\+Socket client or server. There are some minimal convenience features that stray from this (for example the ability to respond to H\+T\+TP requests other than Web\+Socket Upgrades) but these are not the focus of the project. In particular Web\+Socket++ does not intend to implement any non-\/\+Web\+Socket related fallback options (ajax / long polling / comet / etc).

In order to remain compact and improve portability, the Web\+Socket++ project strives to reduce or eliminate external dependencies where possible and appropriate. Web\+Socket++ core has no dependencies other than the C++11 standard library. For non-\/\+C++11 compilers the Boost libraries provide drop in polyfills for the C++11 functionality used.

Web\+Socket++ implements a pluggable data transport component. The default component allows reduced functionality by using S\+TL iostream or raw byte shuffling via reading and writing char buffers. This component has no non-\/\+S\+TL dependencies and can be used in a C++11 environment without Boost. Also included is an Asio based transport component that provides full featured network client/server functionality. This component requires either Boost Asio or a C++11 compiler and standalone Asio. As an advanced option, Web\+Socket++ supports custom transport layers if you want to provide your own using another library.

In order to accommodate the wide variety of use cases Web\+Socket++ has collected, the library is built in a way that most of the major components are loosely coupled and can be swapped out and replaced. Web\+Socket++ will attempt to track the future development of the Web\+Socket protocol and any extensions as they are developed.


\begin{DoxyItemize}
\item \mbox{\hyperlink{getting_started}{Getting Started}}
\item \mbox{\hyperlink{faq}{F\+AQ}}
\item \mbox{\hyperlink{tutorials}{Tutorials}}
\item Change Log / Version History
\item Reference
\begin{DoxyItemize}
\item \mbox{\hyperlink{reference_8handlers}{Handler Reference}} 
\end{DoxyItemize}
\end{DoxyItemize}