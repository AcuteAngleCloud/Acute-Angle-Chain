\href{https://travis-ci.org/google/benchmark}{\tt } \href{https://ci.appveyor.com/project/google/benchmark/branch/master}{\tt } \href{https://coveralls.io/r/google/benchmark}{\tt }

\mbox{\hyperlink{struct_a}{A}} library to support the benchmarking of functions, similar to unit-\/tests.

Discussion group\+: \href{https://groups.google.com/d/forum/benchmark-discuss}{\tt https\+://groups.\+google.\+com/d/forum/benchmark-\/discuss}

I\+RC channel\+: \href{https://freenode.net}{\tt https\+://freenode.\+net} \#googlebenchmark

\href{#known-issues}{\tt Known issues and common problems}

Additional Tooling Documentation

\subsection*{Example usage}

\subsubsection*{Basic usage}

Define a function that executes the code to be measured.


\begin{DoxyCode}
\{c++\}
static void BM\_StringCreation(benchmark::State& state) \{
  while (state.KeepRunning())
    std::string empty\_string;
\}
// Register the function as a benchmark
BENCHMARK(BM\_StringCreation);

// Define another benchmark
static void BM\_StringCopy(benchmark::State& state) \{
  std::string x = "hello";
  while (state.KeepRunning())
    std::string copy(x);
\}
BENCHMARK(BM\_StringCopy);

BENCHMARK\_MAIN();
\end{DoxyCode}


\subsubsection*{Passing arguments}

Sometimes a family of benchmarks can be implemented with just one routine that takes an extra argument to specify which one of the family of benchmarks to run. For example, the following code defines a family of benchmarks for measuring the speed of {\ttfamily memcpy()} calls of different lengths\+:


\begin{DoxyCode}
\{c++\}
static void BM\_memcpy(benchmark::State& state) \{
  char* src = new char[state.range(0)];
  char* dst = new char[state.range(0)];
  memset(src, 'x', state.range(0));
  while (state.KeepRunning())
    memcpy(dst, src, state.range(0));
  state.SetBytesProcessed(int64\_t(state.iterations()) *
                          int64\_t(state.range(0)));
  delete[] src;
  delete[] dst;
\}
BENCHMARK(BM\_memcpy)->Arg(8)->Arg(64)->Arg(512)->Arg(1<<10)->Arg(8<<10);
\end{DoxyCode}


The preceding code is quite repetitive, and can be replaced with the following short-\/hand. The following invocation will pick a few appropriate arguments in the specified range and will generate a benchmark for each such argument.


\begin{DoxyCode}
\{c++\}
BENCHMARK(BM\_memcpy)->Range(8, 8<<10);
\end{DoxyCode}


By default the arguments in the range are generated in multiples of eight and the command above selects \mbox{[} 8, 64, 512, 4k, 8k \mbox{]}. In the following code the range multiplier is changed to multiples of two.


\begin{DoxyCode}
\{c++\}
BENCHMARK(BM\_memcpy)->RangeMultiplier(2)->Range(8, 8<<10);
\end{DoxyCode}
 Now arguments generated are \mbox{[} 8, 16, 32, 64, 128, 256, 512, 1024, 2k, 4k, 8k \mbox{]}.

You might have a benchmark that depends on two or more inputs. For example, the following code defines a family of benchmarks for measuring the speed of set insertion.


\begin{DoxyCode}
\{c++\}
static void BM\_SetInsert(benchmark::State& state) \{
  while (state.KeepRunning()) \{
    state.PauseTiming();
    std::set<int> data = ConstructRandomSet(state.range(0));
    state.ResumeTiming();
    for (int j = 0; j < state.range(1); ++j)
      data.insert(RandomNumber());
  \}
\}
BENCHMARK(BM\_SetInsert)
    ->Args(\{1<<10, 1\})
    ->Args(\{1<<10, 8\})
    ->Args(\{1<<10, 64\})
    ->Args(\{1<<10, 512\})
    ->Args(\{8<<10, 1\})
    ->Args(\{8<<10, 8\})
    ->Args(\{8<<10, 64\})
    ->Args(\{8<<10, 512\});
\end{DoxyCode}


The preceding code is quite repetitive, and can be replaced with the following short-\/hand. The following macro will pick a few appropriate arguments in the product of the two specified ranges and will generate a benchmark for each such pair.


\begin{DoxyCode}
\{c++\}
BENCHMARK(BM\_SetInsert)->Ranges(\{\{1<<10, 8<<10\}, \{1, 512\}\});
\end{DoxyCode}


For more complex patterns of inputs, passing a custom function to {\ttfamily Apply} allows programmatic specification of an arbitrary set of arguments on which to run the benchmark. The following example enumerates a dense range on one parameter, and a sparse range on the second.


\begin{DoxyCode}
\{c++\}
static void CustomArguments(benchmark::internal::Benchmark* b) \{
  for (int i = 0; i <= 10; ++i)
    for (int j = 32; j <= 1024*1024; j *= 8)
      b->Args(\{i, j\});
\}
BENCHMARK(BM\_SetInsert)->Apply(CustomArguments);
\end{DoxyCode}


\subsubsection*{Calculate asymptotic complexity (Big O)}

Asymptotic complexity might be calculated for a family of benchmarks. The following code will calculate the coefficient for the high-\/order term in the running time and the normalized root-\/mean square error of string comparison.


\begin{DoxyCode}
\{c++\}
static void BM\_StringCompare(benchmark::State& state) \{
  std::string s1(state.range(0), '-');
  std::string s2(state.range(0), '-');
  while (state.KeepRunning()) \{
    benchmark::DoNotOptimize(s1.compare(s2));
  \}
  state.SetComplexityN(state.range(0));
\}
BENCHMARK(BM\_StringCompare)
    ->RangeMultiplier(2)->Range(1<<10, 1<<18)->Complexity(benchmark::oN);
\end{DoxyCode}


As shown in the following invocation, asymptotic complexity might also be calculated automatically.


\begin{DoxyCode}
\{c++\}
BENCHMARK(BM\_StringCompare)
    ->RangeMultiplier(2)->Range(1<<10, 1<<18)->Complexity();
\end{DoxyCode}


The following code will specify asymptotic complexity with a lambda function, that might be used to customize high-\/order term calculation.


\begin{DoxyCode}
\{c++\}
BENCHMARK(BM\_StringCompare)->RangeMultiplier(2)
    ->Range(1<<10, 1<<18)->Complexity([](int n)->double\{return n; \});
\end{DoxyCode}


\subsubsection*{Templated benchmarks}

Templated benchmarks work the same way\+: This example produces and consumes messages of size {\ttfamily sizeof(v)} {\ttfamily range\+\_\+x} times. It also outputs throughput in the absence of multiprogramming.


\begin{DoxyCode}
\{c++\}
template <class Q> int BM\_Sequential(benchmark::State& state) \{
  Q q;
  typename Q::value\_type v;
  while (state.KeepRunning()) \{
    for (int i = state.range(0); i--; )
      q.push(v);
    for (int e = state.range(0); e--; )
      q.Wait(&v);
  \}
  // actually messages, not bytes:
  state.SetBytesProcessed(
      static\_cast<int64\_t>(state.iterations())*state.range(0));
\}
BENCHMARK\_TEMPLATE(BM\_Sequential, WaitQueue<int>)->Range(1<<0, 1<<10);
\end{DoxyCode}


\mbox{\hyperlink{struct_three}{Three}} macros are provided for adding benchmark templates.


\begin{DoxyCode}
\{c++\}
#if \_\_cplusplus >= 201103L // C++11 and greater.
#define BENCHMARK\_TEMPLATE(func, ...) // Takes any number of parameters.
#else // C++ < C++11
#define BENCHMARK\_TEMPLATE(func, arg1)
#endif
#define BENCHMARK\_TEMPLATE1(func, arg1)
#define BENCHMARK\_TEMPLATE2(func, arg1, arg2)
\end{DoxyCode}


\subsection*{Passing arbitrary arguments to a benchmark}

In C++11 it is possible to define a benchmark that takes an arbitrary number of extra arguments. The {\ttfamily B\+E\+N\+C\+H\+M\+A\+R\+K\+\_\+\+C\+A\+P\+T\+U\+RE(func, test\+\_\+case\+\_\+name, ...args)} macro creates a benchmark that invokes {\ttfamily func} with the {\ttfamily \mbox{\hyperlink{classbenchmark_1_1_state}{benchmark\+::\+State}}} as the first argument followed by the specified {\ttfamily args...}. The {\ttfamily test\+\_\+case\+\_\+name} is appended to the name of the benchmark and should describe the values passed.

\`{}\`{}{\ttfamily c++ template $<$class ...Extra\+Args$>$} void B\+M\+\_\+takes\+\_\+args(\mbox{\hyperlink{classbenchmark_1_1_state}{benchmark\+::\+State}}\& state, Extra\+Args\&\&... extra\+\_\+args) \{ \mbox{[}...\mbox{]} \} // Registers a benchmark named "B\+M\+\_\+takes\+\_\+args/int\+\_\+string\+\_\+test{\ttfamily that passes // the specified values to}extra\+\_\+args\`{}. B\+E\+N\+C\+H\+M\+A\+R\+K\+\_\+\+C\+A\+P\+T\+U\+RE(B\+M\+\_\+takes\+\_\+args, int\+\_\+string\+\_\+test, 42, std\+::string(\char`\"{}abc\char`\"{})); 
\begin{DoxyCode}
Note that elements of `...args` may refer to global variables. Users should
avoid modifying global state inside of a benchmark.

## Using RegisterBenchmark(name, fn, args...)

The `RegisterBenchmark(name, func, args...)` function provides an alternative
way to create and register benchmarks.
`RegisterBenchmark(name, func, args...)` creates, registers, and returns a
pointer to a new benchmark with the specified `name` that invokes
`func(st, args...)` where `st` is a `benchmark::State` object.

Unlike the `BENCHMARK` registration macros, which can only be used at the global
scope, the `RegisterBenchmark` can be called anywhere. This allows for
benchmark tests to be registered programmatically.

Additionally `RegisterBenchmark` allows any callable object to be registered
as a benchmark. Including capturing lambdas and function objects. This
allows the creation

For Example:
```c++
auto BM\_test = [](benchmark::State& st, auto Inputs) \{ /* ... */ \};

int main(int argc, char** argv) \{
  for (auto& test\_input : \{ /* ... */ \})
      benchmark::RegisterBenchmark(test\_input.name(), BM\_test, test\_input);
  benchmark::Initialize(&argc, argv);
  benchmark::RunSpecifiedBenchmarks();
\}
\end{DoxyCode}


\subsubsection*{Multithreaded benchmarks}

In a multithreaded test (benchmark invoked by multiple threads simultaneously), it is guaranteed that none of the threads will start until all have called {\ttfamily Keep\+Running}, and all will have finished before Keep\+Running returns false. As such, any global setup or teardown can be wrapped in a check against the thread index\+:


\begin{DoxyCode}
\{c++\}
static void BM\_MultiThreaded(benchmark::State& state) \{
  if (state.thread\_index == 0) \{
    // Setup code here.
  \}
  while (state.KeepRunning()) \{
    // Run the test as normal.
  \}
  if (state.thread\_index == 0) \{
    // Teardown code here.
  \}
\}
BENCHMARK(BM\_MultiThreaded)->Threads(2);
\end{DoxyCode}


If the benchmarked code itself uses threads and you want to compare it to single-\/threaded code, you may want to use real-\/time (\char`\"{}wallclock\char`\"{}) measurements for latency comparisons\+:


\begin{DoxyCode}
\{c++\}
BENCHMARK(BM\_test)->Range(8, 8<<10)->UseRealTime();
\end{DoxyCode}


Without {\ttfamily Use\+Real\+Time}, C\+PU time is used by default.

\subsection*{Manual timing}

For benchmarking something for which neither C\+PU time nor real-\/time are correct or accurate enough, completely manual timing is supported using the {\ttfamily Use\+Manual\+Time} function.

When {\ttfamily Use\+Manual\+Time} is used, the benchmarked code must call {\ttfamily Set\+Iteration\+Time} once per iteration of the {\ttfamily Keep\+Running} loop to report the manually measured time.

An example use case for this is benchmarking G\+PU execution (e.\+g. Open\+CL or C\+U\+DA kernels, Open\+GL or Vulkan or Direct3D draw calls), which cannot be accurately measured using C\+PU time or real-\/time. Instead, they can be measured accurately using a dedicated A\+PI, and these measurement results can be reported back with {\ttfamily Set\+Iteration\+Time}.


\begin{DoxyCode}
\{c++\}
static void BM\_ManualTiming(benchmark::State& state) \{
  int microseconds = state.range(0);
  std::chrono::duration<double, std::micro> sleep\_duration \{
    static\_cast<double>(microseconds)
  \};

  while (state.KeepRunning()) \{
    auto start = std::chrono::high\_resolution\_clock::now();
    // Simulate some useful workload with a sleep
    std::this\_thread::sleep\_for(sleep\_duration);
    auto end   = std::chrono::high\_resolution\_clock::now();

    auto elapsed\_seconds =
      std::chrono::duration\_cast<std::chrono::duration<double>>(
        end - start);

    state.SetIterationTime(elapsed\_seconds.count());
  \}
\}
BENCHMARK(BM\_ManualTiming)->Range(1, 1<<17)->UseManualTime();
\end{DoxyCode}


\subsubsection*{Preventing optimisation}

To prevent a value or expression from being optimized away by the compiler the {\ttfamily benchmark\+::\+Do\+Not\+Optimize(...)} and {\ttfamily benchmark\+::\+Clobber\+Memory()} functions can be used.


\begin{DoxyCode}
\{c++\}
static void BM\_test(benchmark::State& state) \{
  while (state.KeepRunning()) \{
      int x = 0;
      for (int i=0; i < 64; ++i) \{
        benchmark::DoNotOptimize(x += i);
      \}
  \}
\}
\end{DoxyCode}


{\ttfamily Do\+Not\+Optimize($<$expr$>$)} forces the {\itshape result} of {\ttfamily $<$expr$>$} to be stored in either memory or a register. For G\+NU based compilers it acts as read/write barrier for global memory. More specifically it forces the compiler to flush pending writes to memory and reload any other values as necessary.

Note that {\ttfamily Do\+Not\+Optimize($<$expr$>$)} does not prevent optimizations on {\ttfamily $<$expr$>$} in any way. {\ttfamily $<$expr$>$} may even be removed entirely when the result is already known. For example\+:


\begin{DoxyCode}
\{c++\}
  /* Example 1: `<expr>` is removed entirely. */
  int foo(int x) \{ return x + 42; \}
  while (...) DoNotOptimize(foo(0)); // Optimized to DoNotOptimize(42);

  /*  Example 2: Result of '<expr>' is only reused */
  int bar(int) \_\_attribute\_\_((const));
  while (...) DoNotOptimize(bar(0)); // Optimized to:
  // int \_\_result\_\_ = bar(0);
  // while (...) DoNotOptimize(\_\_result\_\_);
\end{DoxyCode}


The second tool for preventing optimizations is {\ttfamily Clobber\+Memory()}. In essence {\ttfamily Clobber\+Memory()} forces the compiler to perform all pending writes to global memory. Memory managed by block scope objects must be \char`\"{}escaped\char`\"{} using {\ttfamily Do\+Not\+Optimize(...)} before it can be clobbered. In the below example {\ttfamily Clobber\+Memory()} prevents the call to {\ttfamily v.\+push\+\_\+back(42)} from being optimized away.


\begin{DoxyCode}
\{c++\}
static void BM\_vector\_push\_back(benchmark::State& state) \{
  while (state.KeepRunning()) \{
    std::vector<int> v;
    v.reserve(1);
    benchmark::DoNotOptimize(v.data()); // Allow v.data() to be clobbered.
    v.push\_back(42);
    benchmark::ClobberMemory(); // Force 42 to be written to memory.
  \}
\}
\end{DoxyCode}


Note that {\ttfamily Clobber\+Memory()} is only available for G\+NU or M\+S\+VC based compilers.

\subsubsection*{Set time unit manually}

If a benchmark runs a few milliseconds it may be hard to visually compare the measured times, since the output data is given in nanoseconds per default. In order to manually set the time unit, you can specify it manually\+:


\begin{DoxyCode}
\{c++\}
BENCHMARK(BM\_test)->Unit(benchmark::kMillisecond);
\end{DoxyCode}


\subsection*{Controlling number of iterations}

In all cases, the number of iterations for which the benchmark is run is governed by the amount of time the benchmark takes. Concretely, the number of iterations is at least one, not more than 1e9, until C\+PU time is greater than the minimum time, or the wallclock time is 5x minimum time. The minimum time is set as a flag {\ttfamily -\/-\/benchmark\+\_\+min\+\_\+time} or per benchmark by calling {\ttfamily Min\+Time} on the registered benchmark object.

\subsection*{Reporting the mean and standard devation by repeated benchmarks}

By default each benchmark is run once and that single result is reported. However benchmarks are often noisy and a single result may not be representative of the overall behavior. For this reason it\textquotesingle{}s possible to repeatedly rerun the benchmark.

The number of runs of each benchmark is specified globally by the {\ttfamily -\/-\/benchmark\+\_\+repetitions} flag or on a per benchmark basis by calling {\ttfamily Repetitions} on the registered benchmark object. When a benchmark is run more than once the mean and standard deviation of the runs will be reported.

Additionally the {\ttfamily -\/-\/benchmark\+\_\+report\+\_\+aggregates\+\_\+only=\{true$\vert$false\}} flag or {\ttfamily Report\+Aggregates\+Only(bool)} function can be used to change how repeated tests are reported. By default the result of each repeated run is reported. When this option is \textquotesingle{}true\textquotesingle{} only the mean and standard deviation of the runs is reported. Calling {\ttfamily Report\+Aggregates\+Only(bool)} on a registered benchmark object overrides the value of the flag for that benchmark.

\subsection*{Fixtures}

Fixture tests are created by first defining a type that derives from \mbox{\hyperlink{classbenchmark_1_1_fixture}{benchmark\+::\+Fixture}} and then creating/registering the tests using the following macros\+:


\begin{DoxyItemize}
\item {\ttfamily B\+E\+N\+C\+H\+M\+A\+R\+K\+\_\+\+F(\+Class\+Name, Method)}
\item {\ttfamily B\+E\+N\+C\+H\+M\+A\+R\+K\+\_\+\+D\+E\+F\+I\+N\+E\+\_\+\+F(\+Class\+Name, Method)}
\item {\ttfamily B\+E\+N\+C\+H\+M\+A\+R\+K\+\_\+\+R\+E\+G\+I\+S\+T\+E\+R\+\_\+\+F(\+Class\+Name, Method)}
\end{DoxyItemize}

For Example\+:


\begin{DoxyCode}
\{c++\}
class MyFixture : public benchmark::Fixture \{\};

BENCHMARK\_F(MyFixture, FooTest)(benchmark::State& st) \{
   while (st.KeepRunning()) \{
     ...
  \}
\}

BENCHMARK\_DEFINE\_F(MyFixture, BarTest)(benchmark::State& st) \{
   while (st.KeepRunning()) \{
     ...
  \}
\}
/* BarTest is NOT registered */
BENCHMARK\_REGISTER\_F(MyFixture, BarTest)->Threads(2);
/* BarTest is now registered */
\end{DoxyCode}


\subsection*{User-\/defined counters}

You can add your own counters with user-\/defined names. The example below will add columns \char`\"{}\+Foo\char`\"{}, \char`\"{}\+Bar\char`\"{} and \char`\"{}\+Baz\char`\"{} in its output\+:


\begin{DoxyCode}
\{c++\}
static void UserCountersExample1(benchmark::State& state) \{
  double numFoos = 0, numBars = 0, numBazs = 0;
  while (state.KeepRunning()) \{
    // ... count Foo,Bar,Baz events
  \}
  state.counters["Foo"] = numFoos;
  state.counters["Bar"] = numBars;
  state.counters["Baz"] = numBazs;
\}
\end{DoxyCode}


The {\ttfamily state.\+counters} object is a {\ttfamily \mbox{\hyperlink{classstd_1_1map}{std\+::map}}} with {\ttfamily std\+::string} keys and {\ttfamily \mbox{\hyperlink{class_counter}{Counter}}} values. The latter is a {\ttfamily double}-\/like class, via an implicit conversion to {\ttfamily double\&}. Thus you can use all of the standard arithmetic assignment operators ({\ttfamily =,+=,-\/=,$\ast$=,/=}) to change the value of each counter.

In multithreaded benchmarks, each counter is set on the calling thread only. When the benchmark finishes, the counters from each thread will be summed; the resulting sum is the value which will be shown for the benchmark.

The {\ttfamily \mbox{\hyperlink{class_counter}{Counter}}} constructor accepts two parameters\+: the value as a {\ttfamily double} and a bit flag which allows you to show counters as rates and/or as per-\/thread averages\+:


\begin{DoxyCode}
\{c++\}
  // sets a simple counter
  state.counters["Foo"] = numFoos;

  // Set the counter as a rate. It will be presented divided
  // by the duration of the benchmark.
  state.counters["FooRate"] = Counter(numFoos, benchmark::Counter::kIsRate);

  // Set the counter as a thread-average quantity. It will
  // be presented divided by the number of threads.
  state.counters["FooAvg"] = Counter(numFoos, benchmark::Counter::kAvgThreads);

  // There's also a combined flag:
  state.counters["FooAvgRate"] = Counter(numFoos,benchmark::Counter::kAvgThreadsRate);
\end{DoxyCode}


When you\textquotesingle{}re compiling in C++11 mode or later you can use {\ttfamily insert()} with {\ttfamily std\+::initializer\+\_\+list}\+:


\begin{DoxyCode}
\{c++\}
  // With C++11, this can be done:
  state.counters.insert(\{\{"Foo", numFoos\}, \{"Bar", numBars\}, \{"Baz", numBazs\}\});
  // ... instead of:
  state.counters["Foo"] = numFoos;
  state.counters["Bar"] = numBars;
  state.counters["Baz"] = numBazs;
\end{DoxyCode}


\subsection*{Exiting Benchmarks in Error}

When errors caused by external influences, such as file I/O and network communication, occur within a benchmark the {\ttfamily State\+::\+Skip\+With\+Error(const char$\ast$ msg)} function can be used to skip that run of benchmark and report the error. Note that only future iterations of the {\ttfamily Keep\+Running()} are skipped. Users may explicitly return to exit the benchmark immediately.

The {\ttfamily Skip\+With\+Error(...)} function may be used at any point within the benchmark, including before and after the {\ttfamily Keep\+Running()} loop.

For example\+:


\begin{DoxyCode}
\{c++\}
static void BM\_test(benchmark::State& state) \{
  auto resource = GetResource();
  if (!resource.good()) \{
      state.SkipWithError("Resource is not good!");
      // KeepRunning() loop will not be entered.
  \}
  while (state.KeepRunning()) \{
      auto data = resource.read\_data();
      if (!resource.good()) \{
        state.SkipWithError("Failed to read data!");
        break; // Needed to skip the rest of the iteration.
     \}
     do\_stuff(data);
  \}
\}
\end{DoxyCode}


\subsection*{Running a subset of the benchmarks}

The {\ttfamily -\/-\/benchmark\+\_\+filter=$<$regex$>$} option can be used to only run the benchmarks which match the specified {\ttfamily $<$regex$>$}. For example\+:


\begin{DoxyCode}
$ ./run\_benchmarks.x --benchmark\_filter=BM\_memcpy/32
Run on (1 X 2300 MHz CPU )
2016-06-25 19:34:24
Benchmark              Time           CPU Iterations
----------------------------------------------------
BM\_memcpy/32          11 ns         11 ns   79545455
BM\_memcpy/32k       2181 ns       2185 ns     324074
BM\_memcpy/32          12 ns         12 ns   54687500
BM\_memcpy/32k       1834 ns       1837 ns     357143
\end{DoxyCode}


\subsection*{Output Formats}

The library supports multiple output formats. Use the {\ttfamily -\/-\/benchmark\+\_\+format=$<$console$\vert$json$\vert$csv$>$} flag to set the format type. {\ttfamily console} is the default format.

The Console format is intended to be a human readable format. By default the format generates color output. Context is output on stderr and the tabular data on stdout. Example tabular output looks like\+: 
\begin{DoxyCode}
Benchmark                               Time(ns)    CPU(ns) Iterations
----------------------------------------------------------------------
BM\_SetInsert/1024/1                        28928      29349      23853  133.097kB/s   33.2742k items/s
BM\_SetInsert/1024/8                        32065      32913      21375  949.487kB/s   237.372k items/s
BM\_SetInsert/1024/10                       33157      33648      21431  1.13369MB/s   290.225k items/s
\end{DoxyCode}


The J\+S\+ON format outputs human readable json split into two top level attributes. The {\ttfamily context} attribute contains information about the run in general, including information about the C\+PU and the date. The {\ttfamily benchmarks} attribute contains a list of ever benchmark run. Example json output looks like\+: 
\begin{DoxyCode}
\{
  "context": \{
    "date": "2015/03/17-18:40:25",
    "num\_cpus": 40,
    "mhz\_per\_cpu": 2801,
    "cpu\_scaling\_enabled": false,
    "build\_type": "debug"
  \},
  "benchmarks": [
    \{
      "name": "BM\_SetInsert/1024/1",
      "iterations": 94877,
      "real\_time": 29275,
      "cpu\_time": 29836,
      "bytes\_per\_second": 134066,
      "items\_per\_second": 33516
    \},
    \{
      "name": "BM\_SetInsert/1024/8",
      "iterations": 21609,
      "real\_time": 32317,
      "cpu\_time": 32429,
      "bytes\_per\_second": 986770,
      "items\_per\_second": 246693
    \},
    \{
      "name": "BM\_SetInsert/1024/10",
      "iterations": 21393,
      "real\_time": 32724,
      "cpu\_time": 33355,
      "bytes\_per\_second": 1199226,
      "items\_per\_second": 299807
    \}
  ]
\}
\end{DoxyCode}


The C\+SV format outputs comma-\/separated values. The {\ttfamily context} is output on stderr and the C\+SV itself on stdout. Example C\+SV output looks like\+: 
\begin{DoxyCode}
name,iterations,real\_time,cpu\_time,bytes\_per\_second,items\_per\_second,label
"BM\_SetInsert/1024/1",65465,17890.7,8407.45,475768,118942,
"BM\_SetInsert/1024/8",116606,18810.1,9766.64,3.27646e+06,819115,
"BM\_SetInsert/1024/10",106365,17238.4,8421.53,4.74973e+06,1.18743e+06,
\end{DoxyCode}


\subsection*{Output Files}

The library supports writing the output of the benchmark to a file specified by {\ttfamily -\/-\/benchmark\+\_\+out=$<$filename$>$}. The format of the output can be specified using {\ttfamily -\/-\/benchmark\+\_\+out\+\_\+format=\{json$\vert$console$\vert$csv\}}. Specifying {\ttfamily -\/-\/benchmark\+\_\+out} does not suppress the console output.

\subsection*{Debug vs Release}

By default, benchmark builds as a debug library. You will see a warning in the output when this is the case. To build it as a release library instead, use\+:


\begin{DoxyCode}
cmake -DCMAKE\_BUILD\_TYPE=Release
\end{DoxyCode}


To enable link-\/time optimisation, use


\begin{DoxyCode}
cmake -DCMAKE\_BUILD\_TYPE=Release -DBENCHMARK\_ENABLE\_LTO=true
\end{DoxyCode}


\subsection*{Linking against the library}

When using gcc, it is necessary to link against pthread to avoid runtime exceptions. This is due to how gcc implements std\+::thread. See \href{https://github.com/google/benchmark/issues/67}{\tt issue \#67} for more details.

\subsection*{Compiler Support}

Google Benchmark uses C++11 when building the library. As such we require a modern C++ toolchain, both compiler and standard library.

The following minimum versions are strongly recommended build the library\+:


\begin{DoxyItemize}
\item G\+CC 4.\+8
\item Clang 3.\+4
\item Visual Studio 2013
\item Intel 2015 Update 1
\end{DoxyItemize}

Anything older {\itshape may} work.

Note\+: Using the library and its headers in C++03 is supported. C++11 is only required to build the library.

\section*{Known Issues}

\subsubsection*{Windows}


\begin{DoxyItemize}
\item Users must manually link {\ttfamily shlwapi.\+lib}. Failure to do so may result in unresolved symbols. 
\end{DoxyItemize}