Handlers allow Web\+Socket++ programs to receive notifications about events that happen in relation to their connections. Some handlers also behave as hooks that give the program a chance to modify state or adjust settings before the connection continues.

Handlers are registered by calling the appropriate {\ttfamily set\+\_\+$\ast$\+\_\+handler} method on either an endpoint or connection. The $\ast$ refers to the name of the handler (as specified in the signature field below). For example, to set the open handler, call {\ttfamily set\+\_\+open\+\_\+handler(...)}.

Setting handlers on an endpoint will result in them being copied as the default handler to all new connections created by that endpoint. Changing an endpoint\textquotesingle{}s handlers will not affect connections that are already in progress. This includes connections that are in the listening state. As such, it is important to set any endpoint handlers before you call {\ttfamily endpoint\+::start\+\_\+accept} or else the handlers will not be attached to your first connection.

Setting handlers on a connection will result in the handler being changed for that connection only, starting at the next time that handler is called. This can be used to change the handler during a connection.

\subsection*{Connection Handlers }

These handlers will be called at most once per connection in the order specified below.

\subsubsection*{Socket Init Handler}

\tabulinesep=1mm
\begin{longtabu} spread 0pt [c]{*{3}{|X[-1]}|}
\hline
\rowcolor{\tableheadbgcolor}\textbf{ Event  }&\textbf{ Signature  }&\textbf{ Availability   }\\\cline{1-3}
\endfirsthead
\hline
\endfoot
\hline
\rowcolor{\tableheadbgcolor}\textbf{ Event  }&\textbf{ Signature  }&\textbf{ Availability   }\\\cline{1-3}
\endhead
Socket initialization  &{\ttfamily socket\+\_\+init(connection\+\_\+hdl, asio\+::ip\+::tcp\+::socket\&)}  &0.\+3.\+0 Asio Transport   \\\cline{1-3}
\end{longtabu}


This hook is triggered after the socket has been initialized but before a connection is established. It allows setting arbitrary socket options before connections are sent/recieved.

\subsubsection*{T\+CP Pre-\/init Handler}

\tabulinesep=1mm
\begin{longtabu} spread 0pt [c]{*{3}{|X[-1]}|}
\hline
\rowcolor{\tableheadbgcolor}\textbf{ Event  }&\textbf{ Signature  }&\textbf{ Availability   }\\\cline{1-3}
\endfirsthead
\hline
\endfoot
\hline
\rowcolor{\tableheadbgcolor}\textbf{ Event  }&\textbf{ Signature  }&\textbf{ Availability   }\\\cline{1-3}
\endhead
T\+CP established, no data sent  &{\ttfamily tcp\+\_\+pre\+\_\+init(connection\+\_\+hdl)}  &0.\+3.\+0 Asio Transport   \\\cline{1-3}
\end{longtabu}


This hook is triggered after the T\+CP connection is established, but before any pre-\/\+Web\+Socket-\/handshake operations have been run. Common pre-\/handshake operations include T\+LS handshakes and proxy connections.

\subsubsection*{T\+CP Post-\/init Handler}

\tabulinesep=1mm
\begin{longtabu} spread 0pt [c]{*{3}{|X[-1]}|}
\hline
\rowcolor{\tableheadbgcolor}\textbf{ Event  }&\textbf{ Signature  }&\textbf{ Availability   }\\\cline{1-3}
\endfirsthead
\hline
\endfoot
\hline
\rowcolor{\tableheadbgcolor}\textbf{ Event  }&\textbf{ Signature  }&\textbf{ Availability   }\\\cline{1-3}
\endhead
Request for T\+LS context  &{\ttfamily tls\+\_\+context\+\_\+ptr tls\+\_\+init(connection\+\_\+hdl)}  &0.\+3.\+0 Asio Transport with T\+LS   \\\cline{1-3}
\end{longtabu}


This hook is triggered before the T\+LS handshake to request the T\+LS context to use. You must return a pointer to a configured T\+LS conext to continue. This provides the opportuinity to set up the T\+LS settings, certificates, etc.

\subsubsection*{Validate Handler}

\tabulinesep=1mm
\begin{longtabu} spread 0pt [c]{*{3}{|X[-1]}|}
\hline
\rowcolor{\tableheadbgcolor}\textbf{ Event  }&\textbf{ Signature  }&\textbf{ Availability   }\\\cline{1-3}
\endfirsthead
\hline
\endfoot
\hline
\rowcolor{\tableheadbgcolor}\textbf{ Event  }&\textbf{ Signature  }&\textbf{ Availability   }\\\cline{1-3}
\endhead
Hook to accept or reject a connection  &{\ttfamily bool validate(connection\+\_\+hdl)}  &0.\+3.\+0 Core, Server role only   \\\cline{1-3}
\end{longtabu}


This hook is triggered for servers during the opening handshake after the request has been processed but before the response has been sent. It gives a program the opportunity to inspect headers and other connection details and either accept or reject the connection. Validate happens before the open or fail handler.

Return true to accept the connection, false to reject. If no validate handler is registered, all connections will be accepted.

\subsubsection*{Open Connection Handler}

\tabulinesep=1mm
\begin{longtabu} spread 0pt [c]{*{3}{|X[-1]}|}
\hline
\rowcolor{\tableheadbgcolor}\textbf{ Event  }&\textbf{ Signature  }&\textbf{ Availability   }\\\cline{1-3}
\endfirsthead
\hline
\endfoot
\hline
\rowcolor{\tableheadbgcolor}\textbf{ Event  }&\textbf{ Signature  }&\textbf{ Availability   }\\\cline{1-3}
\endhead
Successful new connection  &{\ttfamily open(connection\+\_\+hdl)}  &0.\+3.\+0 Core   \\\cline{1-3}
\end{longtabu}


Either open or fail will be called for each connection. Never both. All connections that begin with an open handler call will also have a matching close handler call when the connection ends.

\subsubsection*{Fail Connection Handler}

\tabulinesep=1mm
\begin{longtabu} spread 0pt [c]{*{3}{|X[-1]}|}
\hline
\rowcolor{\tableheadbgcolor}\textbf{ Event  }&\textbf{ Signature  }&\textbf{ Availability   }\\\cline{1-3}
\endfirsthead
\hline
\endfoot
\hline
\rowcolor{\tableheadbgcolor}\textbf{ Event  }&\textbf{ Signature  }&\textbf{ Availability   }\\\cline{1-3}
\endhead
Connection failed (before opening)  &{\ttfamily fail(connection\+\_\+hdl)}  &0.\+3.\+0 Core   \\\cline{1-3}
\end{longtabu}


Either open or fail will be called for each connection. Never both. Connections that fail will never have a close handler called.

\subsubsection*{Close Connection Handler}

\tabulinesep=1mm
\begin{longtabu} spread 0pt [c]{*{3}{|X[-1]}|}
\hline
\rowcolor{\tableheadbgcolor}\textbf{ Event  }&\textbf{ Signature  }&\textbf{ Availability   }\\\cline{1-3}
\endfirsthead
\hline
\endfoot
\hline
\rowcolor{\tableheadbgcolor}\textbf{ Event  }&\textbf{ Signature  }&\textbf{ Availability   }\\\cline{1-3}
\endhead
Connection closed (after opening)  &{\ttfamily close(connection\+\_\+hdl)}  &0.\+3.\+0 Core   \\\cline{1-3}
\end{longtabu}


Close will be called exactly once for every connection that open was called for. Close is not called for failed connections.

\subsection*{Message Handlers }

These handers are called in response to incoming messages or message like events. They only will be called while the connection is in the open state.

\subsubsection*{Message Handler}

\tabulinesep=1mm
\begin{longtabu} spread 0pt [c]{*{3}{|X[-1]}|}
\hline
\rowcolor{\tableheadbgcolor}\textbf{ Event  }&\textbf{ Signature  }&\textbf{ Availability   }\\\cline{1-3}
\endfirsthead
\hline
\endfoot
\hline
\rowcolor{\tableheadbgcolor}\textbf{ Event  }&\textbf{ Signature  }&\textbf{ Availability   }\\\cline{1-3}
\endhead
Data message recieved  &{\ttfamily message(connection\+\_\+hdl, message\+\_\+ptr)}  &0.\+3.\+0 Core   \\\cline{1-3}
\end{longtabu}


Applies to all non-\/control messages, including both text and binary opcodes. The {\ttfamily message\+\_\+ptr} type and its A\+PI depends on your endpoint type and its config.

\subsubsection*{Ping Handler}

\tabulinesep=1mm
\begin{longtabu} spread 0pt [c]{*{3}{|X[-1]}|}
\hline
\rowcolor{\tableheadbgcolor}\textbf{ Event  }&\textbf{ Signature  }&\textbf{ Availability   }\\\cline{1-3}
\endfirsthead
\hline
\endfoot
\hline
\rowcolor{\tableheadbgcolor}\textbf{ Event  }&\textbf{ Signature  }&\textbf{ Availability   }\\\cline{1-3}
\endhead
Ping recieved  &{\ttfamily bool ping(connection\+\_\+hdl, std\+::string)}  &0.\+3.\+0 Core   \\\cline{1-3}
\end{longtabu}


Second (string) argument is the binary ping payload. Handler return value indicates whether or not to respond to the ping with a pong. If no ping handler is set, Web\+Socket++ will respond with a pong containing the same binary data as the ping (Per requirements in R\+F\+C6455).

\subsubsection*{Pong Handler}

\tabulinesep=1mm
\begin{longtabu} spread 0pt [c]{*{3}{|X[-1]}|}
\hline
\rowcolor{\tableheadbgcolor}\textbf{ Event  }&\textbf{ Signature  }&\textbf{ Availability   }\\\cline{1-3}
\endfirsthead
\hline
\endfoot
\hline
\rowcolor{\tableheadbgcolor}\textbf{ Event  }&\textbf{ Signature  }&\textbf{ Availability   }\\\cline{1-3}
\endhead
Pong recieved  &{\ttfamily pong(connection\+\_\+hdl, std\+::string)}  &0.\+3.\+0 Core   \\\cline{1-3}
\end{longtabu}


Second (string) argument is the binary pong payload.

\subsubsection*{Pong Timeout Handler}

\tabulinesep=1mm
\begin{longtabu} spread 0pt [c]{*{3}{|X[-1]}|}
\hline
\rowcolor{\tableheadbgcolor}\textbf{ Event  }&\textbf{ Signature  }&\textbf{ Availability   }\\\cline{1-3}
\endfirsthead
\hline
\endfoot
\hline
\rowcolor{\tableheadbgcolor}\textbf{ Event  }&\textbf{ Signature  }&\textbf{ Availability   }\\\cline{1-3}
\endhead
Timed out while waiting for a pong  &{\ttfamily pong\+\_\+timeout(connection\+\_\+hdl, std\+::string)}  &0.\+3.\+0 Core, transport with timer support   \\\cline{1-3}
\end{longtabu}


Triggered if there is no response to a ping after the configured duration. The second (string) argument is the binary payload of the unanswered ping.

\subsubsection*{H\+T\+TP Handler}

\tabulinesep=1mm
\begin{longtabu} spread 0pt [c]{*{3}{|X[-1]}|}
\hline
\rowcolor{\tableheadbgcolor}\textbf{ Event  }&\textbf{ Signature  }&\textbf{ Availability   }\\\cline{1-3}
\endfirsthead
\hline
\endfoot
\hline
\rowcolor{\tableheadbgcolor}\textbf{ Event  }&\textbf{ Signature  }&\textbf{ Availability   }\\\cline{1-3}
\endhead
H\+T\+TP request recieved  &{\ttfamily http(connection\+\_\+hdl}  &0.\+3.\+0 Core, Server role only   \\\cline{1-3}
\end{longtabu}


Called when H\+T\+TP requests that are not Web\+Socket handshake upgrade requests are recieved. Allows responding to regular H\+T\+TP requests. If no handler is registered a 426/\+Upgrade Required error is returned.

\subsubsection*{Interrupt Handler}

\tabulinesep=1mm
\begin{longtabu} spread 0pt [c]{*{3}{|X[-1]}|}
\hline
\rowcolor{\tableheadbgcolor}\textbf{ Event  }&\textbf{ Signature  }&\textbf{ Availability   }\\\cline{1-3}
\endfirsthead
\hline
\endfoot
\hline
\rowcolor{\tableheadbgcolor}\textbf{ Event  }&\textbf{ Signature  }&\textbf{ Availability   }\\\cline{1-3}
\endhead
Connection was manually interrupted  &{\ttfamily interrupt(connection\+\_\+hdl)}  &0.\+3.\+0 Core   \\\cline{1-3}
\end{longtabu}


Interrupt events can be triggered by calling {\ttfamily endpoint\+::interrupt} or {\ttfamily connection\+::interrupt}. Interrupt is similar to a timer event with duration zero but with lower overhead. It is useful for single threaded programs to allow breaking up a very long handler into multiple parts and for multi threaded programs as a way for worker threads to signale to the main/network thread that an event is ready.

todo\+: write low and high watermark handlers 