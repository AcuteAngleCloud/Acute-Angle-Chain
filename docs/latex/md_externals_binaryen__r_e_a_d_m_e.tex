\href{https://travis-ci.org/WebAssembly/binaryen}{\tt } \href{https://ci.appveyor.com/project/WebAssembly/binaryen/branch/master}{\tt }

\section*{Binaryen}

Binaryen is a compiler and toolchain infrastructure library for Web\+Assembly, written in C++. It aims to make \href{https://github.com/WebAssembly/binaryen/wiki/Compiling-to-WebAssembly-with-Binaryen}{\tt compiling to Web\+Assembly} {\bfseries easy, fast, and effective}\+:


\begin{DoxyItemize}
\item {\bfseries Easy}\+: Binaryen has a simple \href{https://github.com/WebAssembly/binaryen/wiki/Compiling-to-WebAssembly-with-Binaryen#c-api-1}{\tt C A\+PI} in a single header. It accepts input in \href{https://github.com/WebAssembly/binaryen/wiki/Compiling-to-WebAssembly-with-Binaryen#what-do-i-need-to-have-in-order-to-use-binaryen-to-compile-to-webassembly}{\tt Web\+Assembly-\/like form} but also accepts a general \href{https://github.com/WebAssembly/binaryen/wiki/Compiling-to-WebAssembly-with-Binaryen#cfg-api}{\tt control flow graph} for compilers that prefer that.
\item {\bfseries Fast}\+: Binaryen\textquotesingle{}s internal IR uses compact data structures and is designed for completely parallel codegen and optimization, using all available C\+PU cores. Binaryen\textquotesingle{}s IR also compiles down to Web\+Assembly extremely easily and quickly because it is essentially a subset of Web\+Assembly.
\item {\bfseries Effective}\+: Binaryen\textquotesingle{}s optimizer has \href{https://github.com/WebAssembly/binaryen/tree/master/src/passes}{\tt many passes} that can improve code very significantly (e.\+g. local coloring to coalesce local variables; dead code elimination; precomputing expressions when possible at compile time; etc.). These optimizations aim to make Binaryen powerful enough to be \href{https://kripken.github.io/talks/binaryen.html#/9}{\tt used as a compiler backend by itself}. \mbox{\hyperlink{struct_one}{One}} specific area of focus is on Web\+Assembly-\/specific optimizations (that general-\/purpose compilers might not do), which you can think of as \href{https://kripken.github.io/talks/binaryen.html#/2}{\tt wasm minification}, similar to minification for Java\+Script, C\+SS, etc., all of which are language-\/specific (an example of such an optimization is block return value generation in {\ttfamily Simplify\+Locals}).
\end{DoxyItemize}

Compilers built using Binaryen include


\begin{DoxyItemize}
\item \href{https://github.com/WebAssembly/binaryen/blob/master/src/asm2wasm.h}{\tt {\ttfamily asm2wasm}} which compiles asm.\+js
\item \href{https://github.com/WebAssembly/binaryen/blob/master/src/s2wasm.h}{\tt {\ttfamily s2wasm}} which compiles the L\+L\+VM Web\+Assembly\textquotesingle{}s backend {\ttfamily .s} output format
\item \href{https://github.com/brson/mir2wasm/}{\tt {\ttfamily mir2wasm}} which compiles Rust M\+IR
\end{DoxyItemize}

Those compilers generate Binaryen IR which can then be optimized and emitted as Web\+Assembly (the first two use the internal C++ A\+PI, the last the \mbox{\hyperlink{struct_c}{C}} A\+PI).

Binaryen also provides a set of {\bfseries toolchain utilities} that can


\begin{DoxyItemize}
\item {\bfseries Parse} and {\bfseries emit} Web\+Assembly. In particular this lets you load Web\+Assembly, optimize it using Binaryen, and re-\/emit it, thus implementing a wasm-\/to-\/wasm optimizer.
\item {\bfseries Interpret} Web\+Assembly as well as run the Web\+Assembly spec tests.
\item Integrate with {\bfseries \href{http://emscripten.org}{\tt Emscripten}} in order to provide a complete compiler toolchain from \mbox{\hyperlink{struct_c}{C}} and C++ to Web\+Assembly.
\item {\bfseries Polyfill} Web\+Assembly by running it in the interpreter compiled to Java\+Script, if the browser does not yet have native support (useful for testing).
\end{DoxyItemize}

Consult the contributing instructions if you\textquotesingle{}re interested in participating.

\subsection*{Binaryen IR}

Binaryen\textquotesingle{}s internal IR is an A\+ST, designed to be


\begin{DoxyItemize}
\item {\bfseries Flexible and fast} for optimization.
\item {\bfseries As close as possible to Web\+Assembly} so it is simple and fast to convert it to and from Web\+Assembly.
\end{DoxyItemize}

The differences between Binaryen IR and Web\+Assembly are\+:


\begin{DoxyItemize}
\item Binaryen IR \href{https://github.com/WebAssembly/binaryen/issues/663}{\tt is an A\+ST}, for convenience of optimization. This differs from the Web\+Assembly binary format which is a stack machine.
\item Web\+Assembly limits block/if/loop types to none and the concrete value types (i32, i64, f32, f64). Binaryen IR has an unreachable type, and it allows block/if/loop to take it, allowing \href{https://github.com/WebAssembly/binaryen/issues/903}{\tt local transforms that don\textquotesingle{}t need to know the global context}.
\item Binaryen IR\textquotesingle{}s text format requires the names of blocks and loops to be unique. This differs from the Web\+Assembly s-\/expression format which allows duplicate names (and depends on scoping to disambiguate).
\end{DoxyItemize}

As a result, you might notice that round-\/trip conversions (wasm =$>$ Binaryen IR =$>$ wasm) change code a little in some corner cases.

\subsection*{Tools}

This repository contains code that builds the following tools in {\ttfamily bin/}\+:


\begin{DoxyItemize}
\item {\bfseries wasm-\/shell}\+: \mbox{\hyperlink{struct_a}{A}} shell that can load and interpret Web\+Assembly code. It can also run the spec test suite.
\item {\bfseries wasm-\/as}\+: Assembles Web\+Assembly in text format (currently S-\/\+Expression format) into binary format (going through Binaryen IR).
\item {\bfseries wasm-\/dis}\+: Un-\/assembles Web\+Assembly in binary format into text format (going through Binaryen IR).
\item {\bfseries wasm-\/opt}\+: Loads Web\+Assembly and runs Binaryen IR passes on it.
\item {\bfseries asm2wasm}\+: An asm.\+js-\/to-\/\+Web\+Assembly compiler, using Emscripten\textquotesingle{}s asm optimizer infrastructure. This is used by Emscripten in Binaryen mode when it uses Emscripten\textquotesingle{}s fastcomp asm.\+js backend.
\item {\bfseries s2wasm}\+: \mbox{\hyperlink{struct_a}{A}} compiler from the {\ttfamily .s} format emitted by the new Web\+Assembly backend being developed in L\+L\+VM. This is used by Emscripten in Binaryen mode when it integrates with the new L\+L\+VM backend.
\item {\bfseries wasm.\+js}\+: wasm.\+js contains Binaryen components compiled to Java\+Script, including the interpreter, {\ttfamily asm2wasm}, the S-\/\+Expression parser, etc., which allow you to use Binaryen with Emscripten and execute code compiled to W\+A\+SM even if the browser doesn\textquotesingle{}t have native support yet. This can be useful as a (slow) polyfill.
\item {\bfseries binaryen.\+js}\+: \mbox{\hyperlink{struct_a}{A}} standalone Java\+Script library that exposes Binaryen methods for \href{https://github.com/WebAssembly/binaryen/blob/master/test/binaryen.js/test.js}{\tt creating and optimizing W\+A\+SM modules}.
\end{DoxyItemize}

Usage instructions for each are below.

\subsection*{Building}


\begin{DoxyCode}
cmake . && make
\end{DoxyCode}
 Note that you can also use {\ttfamily ninja} as your generator\+: {\ttfamily cmake -\/G Ninja . \&\& ninja}


\begin{DoxyItemize}
\item \mbox{\hyperlink{struct_a}{A}} C++11 compiler is required.
\item The Java\+Script components can be built using {\ttfamily build-\/js.\+sh}, see notes inside. Normally this is not needed as builds are provided in this repo already.
\end{DoxyItemize}

If you also want to compile C/\+C++ to Web\+Assembly (and not just asm.\+js to Web\+Assembly), you\textquotesingle{}ll need Emscripten. You\textquotesingle{}ll need the {\ttfamily incoming} branch there (which you can get via \href{http://kripken.github.io/emscripten-site/docs/getting_started/downloads.html}{\tt the S\+DK}), for more details see \href{https://github.com/kripken/emscripten/wiki/WebAssembly}{\tt the wiki}.

\subsection*{Running}

\subsubsection*{wasm-\/opt}

Run


\begin{DoxyCode}
bin/wasm-opt [.wast file] [options] [passes, see --help] [--help]
\end{DoxyCode}


The wasm optimizer receives a .wast file as input, and can run transformation passes on it, as well as print it (before and/or after the transformations). For example, try


\begin{DoxyCode}
bin/wasm-opt test/passes/lower-if-else.wast --print
\end{DoxyCode}


That will pretty-\/print out one of the test cases in the test suite. To run a transformation pass on it, try


\begin{DoxyCode}
bin/wasm-opt test/passes/lower-if-else.wast --print --lower-if-else
\end{DoxyCode}


The {\ttfamily lower-\/if-\/else} pass lowers if-\/else into a block and a break. You can see the change the transformation causes by comparing the output of the two print commands.

It\textquotesingle{}s easy to add your own transformation passes to the shell, just add {\ttfamily .cpp} files into {\ttfamily src/passes}, and rebuild the shell. For example code, take a look at the \href{https://github.com/WebAssembly/binaryen/blob/master/src/passes/LowerIfElse.cpp}{\tt {\ttfamily lower-\/if-\/else} pass}.

Some more notes\+:


\begin{DoxyItemize}
\item See {\ttfamily bin/wasm-\/opt -\/-\/help} for the full list of options and passes.
\item Passing {\ttfamily -\/-\/debug} will emit some debugging info.
\end{DoxyItemize}

\subsubsection*{asm2wasm}

run


\begin{DoxyCode}
bin/asm2wasm [input.asm.js file]
\end{DoxyCode}


This will print out a Web\+Assembly module in s-\/expression format to the console.

For example, try


\begin{DoxyCode}
$ bin/asm2wasm test/hello\_world.asm.js
\end{DoxyCode}


That input file contains


\begin{DoxyCode}
function () \{
  "use asm";
  function add(x, y) \{
    x = x | 0;
    y = y | 0;
    return x + y | 0;
  \}
  return \{ add: add \};
\}
\end{DoxyCode}


You should see something like this\+:



By default you should see pretty colors as in that image. Set {\ttfamily C\+O\+L\+O\+RS=0} in the env to disable colors if you prefer that. On Linux and Mac, you can set {\ttfamily C\+O\+L\+O\+RS=1} in the env to force colors (useful when piping to {\ttfamily more}, for example). For Windows, pretty colors are only available when {\ttfamily stdout/stderr} are not redirected/piped.

Pass {\ttfamily -\/-\/debug} on the command line to see debug info, about asm.\+js functions as they are parsed, etc.

\subsubsection*{C/\+C++ Source ⇒ asm2wasm ⇒ Web\+Assembly}

When using {\ttfamily emcc} with the {\ttfamily B\+I\+N\+A\+R\+Y\+EN} option, it will use Binaryen to build to Web\+Assembly. This lets you compile \mbox{\hyperlink{struct_c}{C}} and C++ to Web\+Assembly, with emscripten using asm.\+js internally as a build step. Since emscripten\textquotesingle{}s asm.\+js generation is very stable, and asm2wasm is a fairly simple process, this method of compiling \mbox{\hyperlink{struct_c}{C}} and C++ to Web\+Assembly is usable already. See the \href{https://github.com/kripken/emscripten/wiki/WebAssembly}{\tt emscripten wiki} for more details about how to use it.

\subsubsection*{C/\+C++ Source ⇒ Web\+Assembly L\+L\+VM backend ⇒ s2wasm ⇒ Web\+Assembly}

Binaryen\textquotesingle{}s {\ttfamily s2wasm} tool can translate the {\ttfamily .s} output from the L\+L\+VM Web\+Assembly backend into Web\+Assembly. You can receive {\ttfamily .s} output from {\ttfamily llc}, and then run {\ttfamily s2wasm} on that\+:


\begin{DoxyCode}
llc code.ll -march=wasm32 -filetype=asm -o code.s
s2wasm code.s > code.wast
\end{DoxyCode}


You can also use Emscripten, which will do those steps for you (as well as link to system libraries, etc.). You can use either normal Emscripten, including it\textquotesingle{}s \char`\"{}fastcomp\char`\"{} fork of L\+L\+VM, or you can use \char`\"{}vanilla\char`\"{} L\+L\+VM, that is, pure upstream L\+L\+VM without Emscripten\textquotesingle{}s additions. With Vanilla L\+L\+VM, you can build with


\begin{DoxyCode}
./emcc input.cpp -s BINARYEN=1
\end{DoxyCode}


With normal Emscripten, you will need to tell it to use the Web\+Assembly backend, since its default is asm.\+js, by setting an env var,


\begin{DoxyCode}
EMCC\_WASM\_BACKEND=1 ./emcc input.cpp -s BINARYEN=1
\end{DoxyCode}


(without the env var, the {\ttfamily B\+I\+N\+A\+R\+Y\+EN} option will make it use the asm.\+js backend, then {\ttfamily asm2wasm}).

For more details, see the \href{https://github.com/kripken/emscripten/wiki/WebAssembly}{\tt emscripten wiki}.

\subsection*{Testing}


\begin{DoxyCode}
./check.py
\end{DoxyCode}


(or {\ttfamily python check.\+py}) will run {\ttfamily wasm-\/shell}, {\ttfamily wasm-\/opt}, {\ttfamily asm2wasm}, {\ttfamily wasm.\+js}, etc. on the testcases in {\ttfamily test/}, and verify their outputs.

It will also run {\ttfamily s2wasm} through the last known good L\+L\+VM output from the \href{https://build.chromium.org/p/client.wasm.llvm/console}{\tt build waterfall}.

The {\ttfamily check.\+py} script supports some options\+:


\begin{DoxyCode}
./check.py [--interpreter=/path/to/interpreter] [TEST1] [TEST2]..
\end{DoxyCode}



\begin{DoxyItemize}
\item If an interpreter is provided, we run the output through it, checking for parse errors.
\item If tests are provided, we run exactly those. If none are provided, we run them all.
\item Some tests require {\ttfamily emcc} or {\ttfamily nodejs} in the path. They will not run if the tool cannot be found, and you\textquotesingle{}ll see a warning.
\item We have tests from upstream in {\ttfamily tests/spec} and {\ttfamily tests/waterfall}, in git submodules. Running {\ttfamily ./check.py} should update those.
\end{DoxyItemize}

\subsection*{Design Principles}


\begin{DoxyItemize}
\item {\bfseries Interned strings for names}\+: It\textquotesingle{}s very convenient to have names on nodes, instead of just numeric indices etc. To avoid most of the performance difference between strings and numeric indices, all strings are interned, which means there is a single copy of each string in memory, string comparisons are just a pointer comparison, etc.
\item {\bfseries Allocate in arenas}\+: Based on experience with other optimizing/transformating toolchains, it\textquotesingle{}s not worth the overhead to carefully track memory of individual nodes. Instead, we allocate all elements of a module in an arena, and the entire arena can be freed when the module is no longer needed.
\end{DoxyItemize}

\subsection*{F\+AQ}


\begin{DoxyItemize}
\item How does {\ttfamily asm2wasm} relate to the new Web\+Assembly backend which is being developed in upstream L\+L\+VM?
\end{DoxyItemize}

This is separate from that. {\ttfamily asm2wasm} focuses on compiling asm.\+js to Web\+Assembly, as emitted by Emscripten\textquotesingle{}s asm.\+js backend. This is useful because while in the long term Emscripten hopes to use the new Web\+Assembly backend, the {\ttfamily asm2wasm} route is a very quick and easy way to generate Web\+Assembly output. It will also be useful for benchmarking the new backend as it progresses.


\begin{DoxyItemize}
\item How about compiling Web\+Assembly to asm.\+js (the opposite direction of {\ttfamily asm2wasm})? Wouldn\textquotesingle{}t that be useful for polyfilling?
\end{DoxyItemize}

Experimentation with this is happening, in {\ttfamily wasm2asm}.

This would be useful, but it is a much harder task, due to some decisions made in Web\+Assembly. For example, Web\+Assembly can have control flow nested inside expressions, which can\textquotesingle{}t directly map to asm.\+js. It could be supported by outlining the code to another function, or to compiling it down into new basic blocks and control-\/flow-\/free instructions, but it is hard to do so in a way that is both fast to do and emits code that is fast to execute. On the other hand, compiling asm.\+js to Web\+Assembly is almost straightforward.

We just have to do more work on {\ttfamily wasm2asm} and see how efficient we can make it.


\begin{DoxyItemize}
\item Can {\ttfamily asm2wasm} compile any asm.\+js code?
\end{DoxyItemize}

Almost. Some decisions made in Web\+Assembly preclude that, for example, there are no global variables. That means that {\ttfamily asm2wasm} has to map asm.\+js global variables onto locations in memory, but then it must know of a safe zone in memory in which to do so, and that information is not directly available in asm.\+js.

{\ttfamily asm2wasm} and {\ttfamily emcc\+\_\+to\+\_\+wasm.\+js.\+sh} do some integration with Emscripten in order to work around these issues, like asking Emscripten to reserve same space for the globals, etc.


\begin{DoxyItemize}
\item Why the weird name for the project?
\end{DoxyItemize}

\char`\"{}\+Binaryen\char`\"{} is a combination of {\bfseries binary} -\/ since Web\+Assembly is a binary format for the web -\/ and {\bfseries Emscripten} -\/ with which it can integrate in order to compile \mbox{\hyperlink{struct_c}{C}} and C++ all the way to Web\+Assembly, via asm.\+js. Binaryen began as Emscripten\textquotesingle{}s Web\+Assembly processing library ({\ttfamily wasm-\/emscripten}).

\char`\"{}\+Binaryen\char`\"{} is pronounced \href{http://www.makinggameofthrones.com/production-diary/2011/2/11/official-pronunciation-guide-for-game-of-thrones.html}{\tt in the same manner} as \char`\"{}\mbox{[}\+Targaryen\mbox{]}(https\+://en.\+wikipedia.\+org/wiki/\+List\+\_\+of\+\_\+\+A\+\_\+\+Song\+\_\+of\+\_\+\+Ice\+\_\+and\+\_\+\+Fire\+\_\+characters\#\+House\+\_\+\+Targaryen)\char`\"{}\+: {\itshape bi-\/\+N\+A\+I\+R-\/ee-\/in}. Or something like that? Anyhow, however Targaryen is correctly pronounced, they should rhyme. Aside from pronunciation, the Targaryen house words, \char`\"{}\+Fire and Blood\char`\"{}, have also inspired Binaryen\textquotesingle{}s\+: \char`\"{}\+Code and Bugs.\char`\"{}


\begin{DoxyItemize}
\item Does it compile under Windows and/or Visual Studio?
\end{DoxyItemize}

Yes, it does. Here\textquotesingle{}s a step-\/by-\/step https\+://github.com/brakmic/brakmic/blob/master/webassembly/\+C\+O\+M\+P\+I\+L\+I\+N\+G\+\_\+\+W\+I\+N32.\+md \char`\"{}tutorial\char`\"{} on how to compile it under {\bfseries Windows 10 x64} with {\bfseries C\+Make} and {\bfseries Visual Studio 2015}. Help would be appreciated on Windows and OS \mbox{\hyperlink{class_x}{X}} as most of the core devs are on Linux. 