The purpose of this tutorial is to demonstrate how to setup a local blockchain that can be used to experiment with smart contracts. The first part of this tutorial will focus on\+:


\begin{DoxyEnumerate}
\item Starting a \mbox{\hyperlink{struct_node}{Node}}
\item Creating a Wallet
\item Creating Accounts
\item Deploying Contracts
\item Interacting with Contracts
\end{DoxyEnumerate}

The second part of this tutorial will walk you through creating and deploying your own contracts.

This tutorial assumes that you have installed A\+A\+C\+IO and that {\ttfamily nodaac} and {\ttfamily claac} are in your path.

\subsection*{Starting a Private Blockchain}

You can start your own single-\/node blockchain with this single command\+:


\begin{DoxyCode}
$ nodaac -e -p aacio --plugin aacio::wallet\_api\_plugin --plugin aacio::chain\_api\_plugin --plugin
       aacio::account\_history\_api\_plugin 
...
aacio generated block 046b9984... #101527 @ 2018-04-01T14:24:58.000 with 0 trxs
aacio generated block 5e527ee2... #101528 @ 2018-04-01T14:24:58.500 with 0 trxs
\end{DoxyCode}


This command sets many flags and loads some optional plugins which we will need for the rest of this tutorial. Assuming everything worked properly, you should see a block generation message every 0.\+5 seconds. ~\newline
 
\begin{DoxyCode}
aacio generated block 046b9984... #101527 @ 2018-04-01T14:24:58.000 with 0 trxs
\end{DoxyCode}


This means your local blockchain is live, producing blocks, and ready to be used.

For more information about the arguments to {\ttfamily nodaac} you can use\+:


\begin{DoxyCode}
nodaac --help
\end{DoxyCode}


\subsection*{Creating a Wallet}

\mbox{\hyperlink{struct_a}{A}} wallet is a repository of private keys necessary to authorize actions on the blockchain. These keys are stored on disk encrypted using a password generated for you. This password should be stored in a secure password manager.


\begin{DoxyCode}
$ claac wallet create
Creating wallet: default
Save password to use in the future to unlock this wallet.
Without password imported keys will not be retrievable.
"PW5JuBXoXJ8JHiCTXfXcYuJabjF9f9UNNqHJjqDVY7igVffe3pXub"
\end{DoxyCode}


For the purpose of this simple development environment, your wallet is being managed by your local {\ttfamily nodaac} via the {\ttfamily \mbox{\hyperlink{classaacio_1_1wallet__api__plugin}{aacio\+::wallet\+\_\+api\+\_\+plugin}}} we enabled when we started {\ttfamily nodaac}. Any time you restart {\ttfamily nodaac} you will have to unlock your wallet before you can use the keys within.


\begin{DoxyCode}
$ claac wallet unlock --password PW5JuBXoXJ8JHiCTXfXcYuJabjF9f9UNNqHJjqDVY7igVffe3pXub
Unlocked: default
\end{DoxyCode}


It is generally not secure to use your password directly on the commandline where it gets logged to your bash history, so you can also unlock in interactive mode\+:


\begin{DoxyCode}
$ claac wallet unlock
password:
\end{DoxyCode}


For security purposes it is generally best to leave your wallet locked when you are not using it. To lock your wallet without shutting down {\ttfamily nodaac} you can do\+:


\begin{DoxyCode}
$ claac wallet lock
Locked: default
\end{DoxyCode}


You will need your wallet unlocked for the rest of this tutorial.

All new blockchains start out with a master key for the sole initial account, {\ttfamily aacio}. To interact with the blockchain you will need to import this initial account\textquotesingle{}s private key into your wallet.

Import the master key for the {\ttfamily aacio} account into your wallet. The master key can be found in the {\ttfamily config.\+ini} file in the config folder for {\ttfamily nodaac}. In this example, the default config folder is used. On Linux systems, this will be in {\ttfamily $\sim$/.local/share/aacio/nod\+Acute\+Angle\+Chain/config} and on Mac\+OS, this will be in {\ttfamily $\sim$/\+Library/\+Application Support/aacio/nod\+Acute\+Angle\+Chain/config}.


\begin{DoxyCode}
$ claac wallet import 5Jgm1N6jp3iNbFM45kPtj66xmbqT9fSuVJgPWfnCSPgQanvW6mJ
imported private key for: AAC84snobCGtpQvTTfVaxvMuxGDd4p2LhtQhbWX1yWAU5yt4tiB55
\end{DoxyCode}


\subsection*{Load the Bios Contract}

Now that we have a wallet with the key for the {\ttfamily aacio} account loaded, we can set a default system contract. For the purposes of development, the default {\ttfamily \mbox{\hyperlink{classaacio_1_1bios}{aacio.\+bios}}} contract can be used. This contract enables you to have direct control over the resource allocation of other accounts and to access other privileged A\+PI calls. In a public blockchain, this contract will manage the staking and unstaking of tokens to reserve bandwidth for C\+PU and network activity, and memory for contracts.

The {\ttfamily \mbox{\hyperlink{classaacio_1_1bios}{aacio.\+bios}}} contract can be found in the {\ttfamily contracts/aacio.\+bios} folder of your A\+A\+C\+IO source code. The command sequence below assumes it is being executed from the root of the A\+A\+C\+IO source, but you can execute it from anywhere by specifying the full path to {\ttfamily \$\{A\+A\+C\+I\+O\+\_\+\+S\+O\+U\+R\+CE\}/build/contracts/aacio.bios}.


\begin{DoxyCode}
$ claac set contract aacio build/contracts/aacio.bios -p aacio
Reading WAST...
Assembling WASM...
Publishing contract...
executed transaction: 414cf0dc7740d22474992779b2416b0eabdbc91522c16521307dd682051af083  4068 bytes  10000
       cycles
#         aacio <= aacio::setcode              
       \{"account":"aacio","vmtype":0,"vmversion":0,"code":"0061736d0100000001ab011960037f7e7f0060057f7e7e7e...
#         aacio <= aacio::setabi               
       \{"account":"aacio","abi":\{"types":[],"structs":[\{"name":"set\_account\_limits","base":"","fields":[\{"n...
\end{DoxyCode}


The result of this command sequence is that {\ttfamily claac} generated a transaction with two actions, {\ttfamily aacio\+::setcode} and {\ttfamily aacio\+::setabi}. ~\newline
 The code defines how the contract runs and the abi describes how to convert between binary and json representations of the arguments. While an abi is technically optional, all of the A\+A\+C\+IO tooling depends upon it for ease of use. ~\newline
 Any time you execute a transaction you will see output like\+: 
\begin{DoxyCode}
executed transaction: 414cf0dc7740d22474992779b2416b0eabdbc91522c16521307dd682051af083  4068 bytes  10000
       cycles
#         aacio <= aacio::setcode              
       \{"account":"aacio","vmtype":0,"vmversion":0,"code":"0061736d0100000001ab011960037f7e7f0060057f7e7e7e...
#         aacio <= aacio::setabi               
       \{"account":"aacio","abi":\{"types":[],"structs":[\{"name":"set\_account\_limits","base":"","fields":[\{"n...
\end{DoxyCode}


This can be read as\+: The action {\ttfamily setcode} as defined by {\ttfamily aacio} was executed by {\ttfamily aacio} contract with {\ttfamily \{args...\}}.


\begin{DoxyCode}
#         $\{executor\} <= $\{contract\}:$\{action\} $\{args...\}
> console output from this execution, if any
\end{DoxyCode}


As we will see in a bit, actions can be processed by more than one contract.

The last argument to this call was {\ttfamily -\/p aacio}. This tells {\ttfamily claac} to sign this action with the active authority of the {\ttfamily aacio} account, i.\+e., to sign the action using the private key for the {\ttfamily aacio} account that we imported earlier.

\subsection*{Create an Account}

Now that we have setup the basic system contract, we can start to create our own accounts. We will create two accounts, {\ttfamily user} and {\ttfamily tester}, and we will need to associate a key with each account. In this example, the same key will be used for both accounts.

To do this we first generate a key for the accounts.


\begin{DoxyCode}
$ claac create key
Private key: 5Jmsawgsp1tQ3GD6JyGCwy1dcvqKZgX6ugMVMdjirx85iv5VyPR
Public key: AAC7ijWCBmoXBi3CgtK7DJxentZZeTkeUnaSDvyro9dq7Sd1C3dC4
\end{DoxyCode}


Then we import this key into our wallet\+: 
\begin{DoxyCode}
$ claac wallet import 5Jmsawgsp1tQ3GD6JyGCwy1dcvqKZgX6ugMVMdjirx85iv5VyPR
imported private key for: AAC7ijWCBmoXBi3CgtK7DJxentZZeTkeUnaSDvyro9dq7Sd1C3dC4
\end{DoxyCode}
 {\bfseries N\+O\+TE\+:} Be sure to use the actual key value generated by the {\ttfamily claac} command and not the one shown in the example above!

Keys are not automatically added to a wallet, so skipping this step could result in losing control of your account.

\subsection*{Create \mbox{\hyperlink{struct_two}{Two}} User Accounts}

Next we will create two accounts, {\ttfamily user} and {\ttfamily tester}, using the key we created and imported above.


\begin{DoxyCode}
$ claac create account aacio user AAC7ijWCBmoXBi3CgtK7DJxentZZeTkeUnaSDvyro9dq7Sd1C3dC4
       AAC7ijWCBmoXBi3CgtK7DJxentZZeTkeUnaSDvyro9dq7Sd1C3dC4
executed transaction: 8aedb926cc1ca31642ada8daf4350833c95cbe98b869230f44da76d70f6d6242  364 bytes  1000
       cycles
#         aacio <= aacio::newaccount           
       \{"creator":"aacio","name":"user","owner":\{"threshold":1,"keys":[\{"key":"AAC7ijWCBmoXBi3CgtK7DJxentZZ...

$ claac create account aacio tester AAC7ijWCBmoXBi3CgtK7DJxentZZeTkeUnaSDvyro9dq7Sd1C3dC4
       AAC7ijWCBmoXBi3CgtK7DJxentZZeTkeUnaSDvyro9dq7Sd1C3dC4
executed transaction: 414cf0dc7740d22474992779b2416b0eabdbc91522c16521307dd682051af083 366 bytes  1000
       cycles
#         aacio <= aacio::newaccount           
       \{"creator":"aacio","name":"tester","owner":\{"threshold":1,"keys":[\{"key":"AAC7ijWCBmoXBi3CgtK7DJxentZZ...
\end{DoxyCode}
 {\bfseries N\+O\+TE\+:} The {\ttfamily create account} subcommand requires two keys, one for the Owner\+Key (which in a production environment should be kept highly secure) and one for the Active\+Key. In this tutorial example, the same key is used for both.

Because we are using the {\ttfamily \mbox{\hyperlink{classaacio_1_1account__history__api__plugin}{aacio\+::account\+\_\+history\+\_\+api\+\_\+plugin}}} we can query all accounts that are controlled by our key\+:


\begin{DoxyCode}
$ claac get accounts AAC7ijWCBmoXBi3CgtK7DJxentZZeTkeUnaSDvyro9dq7Sd1C3dC4
\{
  "account\_names": [
    "tester",
    "user"
  ]
\}
\end{DoxyCode}


\subsection*{Create Token Contract}

At this stage the blockchain doesn\textquotesingle{}t do much, so let\textquotesingle{}s deploy the {\ttfamily \mbox{\hyperlink{classaacio_1_1token}{aacio.\+token}}} contract. This contract enables the creation of many different tokens all running on the same contract but potentially managed by different users.

Before we can deploy the token contract we must create an account to deploy it to.


\begin{DoxyCode}
$ claac create account aacio aacio.token  AAC7ijWCBmoXBi3CgtK7DJxentZZeTkeUnaSDvyro9dq7Sd1C3dC4
       AAC7ijWCBmoXBi3CgtK7DJxentZZeTkeUnaSDvyro9dq7Sd1C3dC4
...
\end{DoxyCode}


Then we can deploy the contract which can be found in {\ttfamily \$\{A\+A\+C\+I\+O\+\_\+\+S\+O\+U\+R\+CE\}/build/contracts/aacio.token}


\begin{DoxyCode}
$ claac set contract aacio.token build/contracts/aacio.token -p aacio.token
Reading WAST...
Assembling WASM...
Publishing contract...
executed transaction: 528bdbce1181dc5fd72a24e4181e6587dace8ab43b2d7ac9b22b2017992a07ad  8708 bytes  10000
       cycles
#         aacio <= aacio::setcode              
       \{"account":"aacio.token","vmtype":0,"vmversion":0,"code":"0061736d0100000001ce011d60067f7e7f7f7f7f00...
#         aacio <= aacio::setabi               
       \{"account":"aacio.token","abi":\{"types":[],"structs":[\{"name":"transfer","base":"","fields":[\{"name"...
\end{DoxyCode}


\subsubsection*{Create the A\+AC Token}

You can view the interface to {\ttfamily \mbox{\hyperlink{classaacio_1_1token}{aacio.\+token}}} as defined by {\ttfamily contracts/aacio.\+token/aacio.\mbox{\hyperlink{token_8hpp}{token.\+hpp}}}\+: 
\begin{DoxyCode}
void create( account\_name issuer,
             asset        maximum\_supply,
             uint8\_t      can\_freeze,
             uint8\_t      can\_recall,
             uint8\_t      can\_whitelist );


void issue( account\_name to, asset quantity, string memo );

void transfer( account\_name from,
               account\_name to,
               asset        quantity,
               string       memo );
\end{DoxyCode}


To create a new token we must call the {\ttfamily create(...)} action with the proper arguments. This command will use the symbol of the maximum supply to uniquely identify this token from other tokens. The issuer will be the one with authority to call issue and or perform other actions such as freezing, recalling, and whitelisting of owners.

The concise way to call this method, using positional arguments\+: 
\begin{DoxyCode}
$ claac push action aacio.token create '[ "aacio", "1000000000.0000 AAC", 0, 0, 0]' -p aacio.token
executed transaction: 0e49a421f6e75f4c5e09dd738a02d3f51bd18a0cf31894f68d335cd70d9c0e12  260 bytes  1000
       cycles
#   aacio.token <= aacio.token::create          \{"issuer":"aacio","maximum\_supply":"1000000000.0000
       AAC","can\_freeze":0,"can\_recall":0,"can\_whitelis...
\end{DoxyCode}


Alternatively, a more verbose way to call this method, using named arguments\+:


\begin{DoxyCode}
$ claac push action aacio.token create '\{"issuer":"aacio", "maximum\_supply":"1000000000.0000 AAC",
       "can\_freeze":0, "can\_recall":0, "can\_whitelist":0\}' -p aacio.token
executed transaction: 0e49a421f6e75f4c5e09dd738a02d3f51bd18a0cf31894f68d335cd70d9c0e12  260 bytes  1000
       cycles
#   aacio.token <= aacio.token::create          \{"issuer":"aacio","maximum\_supply":"1000000000.0000
       AAC","can\_freeze":0,"can\_recall":0,"can\_whitelis...
\end{DoxyCode}


This command created a new token {\ttfamily A\+AC} with a pecision of 4 decimials and a maximum supply of 1000000000.\+0000 A\+AC.

In order to create this token we required the permission of the {\ttfamily \mbox{\hyperlink{classaacio_1_1token}{aacio.\+token}}} contract because it \char`\"{}owns\char`\"{} the symbol namespace (e.\+g. \char`\"{}\+A\+A\+C\char`\"{}). Future versions of this contract may allow other parties to buy symbol names automatically. For this reason we must pass {\ttfamily -\/p \mbox{\hyperlink{classaacio_1_1token}{aacio.\+token}}} to authorize this call.

\subsubsection*{Issue Tokens to Account \char`\"{}\+User\char`\"{}}

Now that we have created the token, the issuer can issue new tokens to the account {\ttfamily user} we created earlier.

We will use the positional calling convention (vs named args).


\begin{DoxyCode}
$ claac push action aacio.token issue '[ "user", "100.0000 AAC", "memo" ]' -p aacio
executed transaction: 822a607a9196112831ecc2dc14ffb1722634f1749f3ac18b73ffacd41160b019  268 bytes  1000
       cycles
#   aacio.token <= aacio.token::issue           \{"to":"user","quantity":"100.0000 AAC","memo":"memo"\}
>> issue
#   aacio.token <= aacio.token::transfer        \{"from":"aacio","to":"user","quantity":"100.0000
       AAC","memo":"memo"\}
>> transfer
#         aacio <= aacio.token::transfer        \{"from":"aacio","to":"user","quantity":"100.0000
       AAC","memo":"memo"\}
#          user <= aacio.token::transfer        \{"from":"aacio","to":"user","quantity":"100.0000
       AAC","memo":"memo"\}
\end{DoxyCode}


This time the output contains several different actions\+: one issue and three transfers. While the only action we signed was {\ttfamily issue}, the {\ttfamily issue} action performed an \char`\"{}inline transfer\char`\"{} and the \char`\"{}inline transfer\char`\"{} notified the sender and receiver accounts. The output indicates all of the action handlers that were called, the order they were called in, and whether or not any output was generated by the action.

Technically, the {\ttfamily \mbox{\hyperlink{classaacio_1_1token}{aacio.\+token}}} contract could have skipped the {\ttfamily inline transfer} and opted to just modify the balances directly. However, in this case, the {\ttfamily \mbox{\hyperlink{classaacio_1_1token}{aacio.\+token}}} contract is following our token convention that requires that all account balances be derivable by the sum of the transfer actions that reference them. It also requires that the sender and receiver of funds be notified so they can automate handling deposits and withdrawals.

If you want to see the actual transaction that was broadcast, you can use the {\ttfamily -\/d -\/j} options to indicate \char`\"{}don\textquotesingle{}t broadcast\char`\"{} and \char`\"{}return transaction as json\char`\"{}.


\begin{DoxyCode}
$ claac push action aacio.token issue '["user", "100.0000 AAC", "memo"]' -p aacio -d -j
\{
  "expiration": "2018-04-01T15:20:44",
  "region": 0,
  "ref\_block\_num": 42580,
  "ref\_block\_prefix": 3987474256,
  "net\_usage\_words": 21,
  "kcpu\_usage": 1000,
  "delay\_sec": 0,
  "context\_free\_actions": [],
  "actions": [\{
      "account": "aacio.token",
      "name": "issue",
      "authorization": [\{
          "actor": "aacio",
          "permission": "active"
        \}
      ],
      "data": "00000000007015d640420f000000000004454f5300000000046d656d6f"
    \}
  ],
  "signatures": [
    "AACJzPywCKsgBitRh9kxFNeMJc8BeD6QZLagtXzmdS2ib5gKTeELiVxXvcnrdRUiY3ExP9saVkdkzvUNyRZSXj2CLJnj7U42H"
  ],
  "context\_free\_data": []
\}
\end{DoxyCode}


\subsubsection*{Transfer Tokens to Account \char`\"{}\+Tester\char`\"{}}

Now that account {\ttfamily user} has tokens, we will transfer some to account {\ttfamily tester}. We indicate that {\ttfamily user} authorized this action using the permission argument {\ttfamily -\/p user}.


\begin{DoxyCode}
$ claac push action aacio.token transfer '[ "user", "tester", "25.0000 AAC", "m" ]' -p user
executed transaction: 06d0a99652c11637230d08a207520bf38066b8817ef7cafaab2f0344aafd7018  268 bytes  1000
       cycles
#   aacio.token <= aacio.token::transfer        \{"from":"user","to":"tester","quantity":"25.0000
       AAC","memo":"m"\}
>> transfer
#          user <= aacio.token::transfer        \{"from":"user","to":"tester","quantity":"25.0000
       AAC","memo":"m"\}
#        tester <= aacio.token::transfer        \{"from":"user","to":"tester","quantity":"25.0000
       AAC","memo":"m"\}
\end{DoxyCode}


\subsection*{Hello World Contract}

We will now create our first \char`\"{}hello world\char`\"{} contract. Create a new folder called \char`\"{}hello\char`\"{}, cd into the folder, then create a file \char`\"{}hello.\+cpp\char`\"{} with the following contents\+:

\#\#\#\# hello/hello.\+cpp 
\begin{DoxyCode}
#include <aaciolib/aacio.hpp>
#include <aaciolib/print.hpp>
using namespace aacio;

class hello : public aacio::contract \{
  public:
      using contract::contract;

      /// @abi action 
      void hi( account\_name user ) \{
         print( "Hello, ", name\{user\} );
      \}
\};

AACIO\_ABI( hello, (hi) )
\end{DoxyCode}


You can compile your code to web assmebly (.wast) as follows\+: 
\begin{DoxyCode}
$ aaciocpp -o hello.wast hello.cpp
\end{DoxyCode}
 {\bfseries N\+O\+TE\+:} The compiler might generate warnings. These can be safely ignored.

Now generate the abi\+:


\begin{DoxyCode}
$ aaciocpp -g hello.abi hello.cpp
Generated hello.abi
\end{DoxyCode}


Create an account and upload the contract\+:


\begin{DoxyCode}
$ claac create account aacio hello.code AAC7ijWCBmoXBi3CgtK7DJxentZZeTkeUnaSDvyro9dq7Sd1C3dC4
       AAC7ijWCBmoXBi3CgtK7DJxentZZeTkeUnaSDvyro9dq7Sd1C3dC4
...
$ claac set contract hello.code ../hello -p hello.code
...
\end{DoxyCode}


Now we can run the contract\+:


\begin{DoxyCode}
$ claac push action hello.code hi '["user"]' -p user
executed transaction: 4c10c1426c16b1656e802f3302677594731b380b18a44851d38e8b5275072857  244 bytes  1000
       cycles
#    hello.code <= hello.code::hi               \{"user":"user"\}
>> Hello, user
\end{DoxyCode}


At this time the contract allows anyone to authorize it, we could also say\+:


\begin{DoxyCode}
$ claac push action hello.code hi '["user"]' -p tester
executed transaction: 28d92256c8ffd8b0255be324e4596b7c745f50f85722d0c4400471bc184b9a16  244 bytes  1000
       cycles
#    hello.code <= hello.code::hi               \{"user":"user"\}
>> Hello, user
\end{DoxyCode}


In this case tester is the one who authorized it and user is just an argument. If we want our contact to authenticate the user we are saying \char`\"{}hi\char`\"{} to, then we need to modify the contract to require authentication.

Modify the hi() function in hello.\+cpp as follows\+: 
\begin{DoxyCode}
void hi( account\_name user ) \{
   require\_auth( user );
   print( "Hello, ", name\{user\} );
\}
\end{DoxyCode}
 Repeat the steps to compile the wast file and generate the abi, then set the contract again to deploy the update.

Now if we attempt to mismatch the user and the authority, the contract will throw an error\+: 
\begin{DoxyCode}
$ claac push action hello.code hi '["tester"]' -p user
Error 3030001: missing required authority
Ensure that you have the related authority inside your transaction!;
If you are currently using 'claac push action' command, try to add the relevant authority using -p option.
Error Details:
missing authority of tester
\end{DoxyCode}


We can fix this by giving the permission of tester\+:


\begin{DoxyCode}
$ claac push action hello.code hi '["tester"]' -p tester
executed transaction: 235bd766c2097f4a698cfb948eb2e709532df8d18458b92c9c6aae74ed8e4518  244 bytes  1000
       cycles
#    hello.code <= hello.code::hi               \{"user":"tester"\}
>> Hello, tester
\end{DoxyCode}


\subsection*{Deploy Exchange Contract}

Similar to the examples shown above, we can deploy the exchange contract. It is assumed this is being run from the root of the A\+A\+C\+IO source.


\begin{DoxyCode}
$ claac create account aacio exchange  AAC7ijWCBmoXBi3CgtK7DJxentZZeTkeUnaSDvyro9dq7Sd1C3dC4
       AAC7ijWCBmoXBi3CgtK7DJxentZZeTkeUnaSDvyro9dq7Sd1C3dC4
executed transaction: 4d38de16631a2dc698f1d433f7eb30982d855219e7c7314a888efbbba04e571c  364 bytes  1000
       cycles
#         aacio <= aacio::newaccount           
       \{"creator":"aacio","name":"exchange","owner":\{"threshold":1,"keys":[\{"key":"AAC7ijWCBmoXBi3CgtK7DJxe...

$ claac set contract exchange build/contracts/exchange -p exchange
Reading WAST...
Assembling WASM...
Publishing contract...
executed transaction: 5a63b4de8a1da415590778f163c5ed26dc164c960185b20fd834c297cf7fa8f4  35172 bytes  10000
       cycles
#         aacio <= aacio::setcode              
       \{"account":"exchange","vmtype":0,"vmversion":0,"code":"0061736d0100000001f0023460067f7e7f7f7f7f00600...
#         aacio <= aacio::setabi               
       \{"account":"exchange","abi":\{"types":[\{"new\_type\_name":"account\_name","type":"name"\}],"structs":[\{"n...
\end{DoxyCode}
 