\subsection*{Introduction }

This tutorial provides a step by step discussion of building a basic Web\+Socket++ server. The final product of this tutorial is the \mbox{\hyperlink{classutility__server}{utility\+\_\+server}} example application from the example section. This server demonstrates the following features\+:


\begin{DoxyItemize}
\item Use Asio Transport for networking
\item Accept multiple Web\+Socket connections at once
\item Read incoming messages and perform a few basic actions (echo, broadcast, telemetry, server commands) based on the path
\item Use validate handler to reject connections to invalid paths
\item Serve basic H\+T\+TP responses with the http handler
\item Gracefully exit the server
\item Encrypt connections with T\+LS
\end{DoxyItemize}

This tutorial is current as of the 0.\+6.\+x version of the library.

\subsection*{Chapter 1\+: Initial Setup \& Basics }

\subsubsection*{Step 1}

{\itshape Add Web\+Socket++ includes and set up a a server endpoint type.}

Web\+Socket++ includes two major object types. The endpoint and the connection. The endpoint creates and launches new connections and maintains default settings for those connections. Endpoints also manage any shared network resources.

The connection stores information specific to each Web\+Socket session.

\begin{quote}
{\bfseries Note\+:} Once a connection is launched, there is no link between the endpoint and the connection. All default settings are copied into the new connection by the endpoint. Changing default settings on an endpoint will only affect future connections. \end{quote}
Connections do not maintain a link back to their associated endpoint. Endpoints do not maintain a list of outstanding connections. If your application needs to iterate over all connections it will need to maintain a list of them itself.

Web\+Socket++ endpoints are built by combining an endpoint role with an endpoint config. There are two different types of endpoint roles, one each for the client and server roles in a Web\+Socket session. This is a server tutorial so we will use the server role {\ttfamily \mbox{\hyperlink{classwebsocketpp_1_1server}{websocketpp\+::server}}} which is provided by the {\ttfamily $<$\mbox{\hyperlink{server_8hpp_source}{websocketpp/server.\+hpp}}$>$} header.

\begin{quote}
\paragraph*{Terminology\+: Endpoint Config}

Web\+Socket++ endpoints have a group of settings that may be configured at compile time via the {\ttfamily config} template parameter. \mbox{\hyperlink{struct_a}{A}} config is a struct that contains types and static constants that are used to produce an endpoint with specific properties. Depending on which config is being used the endpoint will have different methods available and may have additional third party dependencies. \end{quote}


The endpoint role takes a template parameter called {\ttfamily config} that is used to configure the behavior of endpoint at compile time. For this example we are going to use a default config provided by the library called {\ttfamily asio}, provided by {\ttfamily $<$\mbox{\hyperlink{asio__no__tls_8hpp_source}{websocketpp/config/asio\+\_\+no\+\_\+tls.\+hpp}}$>$}. This is a server config that uses the Asio library to provide network transport and does not support T\+LS based security. Later on we will discuss how to introduce T\+LS based security into a Web\+Socket++ application, more about the other stock configs, and how to build your own custom configs.

Combine a config with an endpoint role to produce a fully configured endpoint. This type will be used frequently so I would recommend a typedef here.

{\ttfamily typedef \mbox{\hyperlink{classwebsocketpp_1_1server}{websocketpp\+::server}}$<$\mbox{\hyperlink{structwebsocketpp_1_1config_1_1asio}{websocketpp\+::config\+::asio}}$>$ server}

\paragraph*{{\ttfamily \mbox{\hyperlink{classutility__server}{utility\+\_\+server}}} constructor}

This endpoint type will be the base of the \mbox{\hyperlink{classutility__server}{utility\+\_\+server}} object that will keep track of the state of the server. Within the {\ttfamily \mbox{\hyperlink{classutility__server}{utility\+\_\+server}}} constructor several things happen\+:

First, we adjust the endpoint logging behavior to include all error logging channels and all access logging channels except the frame payload, which is particularly noisy and generally useful only for debugging. \mbox{[}T\+O\+DO\+: link to more information about logging\mbox{]}


\begin{DoxyCode}
m\_endpoint.set\_error\_channels(\mbox{\hyperlink{structwebsocketpp_1_1log_1_1elevel_a9b31ff708c221d314f9f4eb3ff2b1ad7}{websocketpp::log::elevel::all}});
m\_endpoint.set\_access\_channels(\mbox{\hyperlink{structwebsocketpp_1_1log_1_1alevel_a853aa0b8976e53f3181af3bc398d493e}{websocketpp::log::alevel::all}} ^ 
      \mbox{\hyperlink{structwebsocketpp_1_1log_1_1alevel_aa38cfdf7a82f33cac319438462707e90}{websocketpp::log::alevel::frame\_payload}});
\end{DoxyCode}


Next, we initialize the transport system underlying the endpoint. This method is specific to the Asio transport not Web\+Socket++ core. It will not be necessary or present in endpoints that use a non-\/asio config.

\begin{quote}
{\bfseries Note\+:} This example uses an internal Asio {\ttfamily io\+\_\+service} that is managed by the endpoint itself. This is a simple arrangement suitable for programs where Web\+Socket++ is the only code using Asio. If you have an existing program that already manages an {\ttfamily io\+\_\+service} object or want to build a new program where Web\+Socket++ handlers share an io\+\_\+service with other handlers you can pass the {\ttfamily io\+\_\+service} you want Web\+Socket++ to register its handlers on to the {\ttfamily init\+\_\+asio()} method and it will use it instead of generating and managing its own. \mbox{[}T\+O\+DO\+: F\+AQ link instead?\mbox{]} \end{quote}



\begin{DoxyCode}
m\_endpoint.init\_asio();
\end{DoxyCode}


\paragraph*{{\ttfamily utility\+\_\+server\+::run} method}

In addition to the constructor, we also add a run method that sets up the listening socket, begins accepting connections, starts the Asio io\+\_\+service event loop.


\begin{DoxyCode}
\textcolor{comment}{// Listen on port 9002}
m\_endpoint.listen(9002);

\textcolor{comment}{// Queues a connection accept operation}
m\_endpoint.start\_accept();

\textcolor{comment}{// Start the Asio io\_service run loop}
m\_endpoint.run();
\end{DoxyCode}


The final line, {\ttfamily m\+\_\+endpoint.\+run();}, will block until the endpoint is instructed to stop listening for new connections. While running it will listen for and process new connections as well as accept and process new data and control messages for existing connections. Web\+Socket++ uses Asio in an asyncronous mode where multiple connections can be similtaneously serviced efficiently within a single thread.

\paragraph*{Build}

Adding Web\+Socket++ has added a few dependencies to our program that must be addressed in the build system. Firstly, the Web\+Socket++ library headers need must be in the include search path of your build system. How exactly this is done depends on where you have the Web\+Socket++ headers installed what build system you are using.

For the rest of this tutorial we are going to assume a C++11 build environment. Web\+Socket++ will work with pre-\/\+C++11 systems if your build system has access to a recent version of the Boost library headers.

Finally, to use the Asio transport config we need to bring in the Asio library. There are two options here. If you have access to a C++11 build environment the standalone version from \href{http://think-async.com}{\tt http\+://think-\/async.\+com} is a good option. This header only library does not bring in any special dependencies and ensures you have the latest version of Asio. If you do not have a C++11 build environment or already have brought in the Boost libraries you can also use the version of Asio bundled with Boost.

To use standalone Asio, make sure the Asio headers are in your include path and define A\+S\+I\+O\+\_\+\+S\+T\+A\+N\+D\+A\+L\+O\+NE. To use Boost Asio, make sure the Boost headers are in your include path and that you are linking to the boost\+\_\+system library.

{\ttfamily c++ -\/std=c++11 step1.\+cpp} (Asio Standalone) OR {\ttfamily c++ -\/std=c++11 step1.\+cpp -\/lboost\+\_\+system} (Boost Asio)

\#\#\#\# Code so far 
\begin{DoxyCode}
\textcolor{comment}{// The ASIO\_STANDALONE define is necessary to use the standalone version of Asio.}
\textcolor{comment}{// Remove if you are using Boost Asio.}
\textcolor{preprocessor}{#define ASIO\_STANDALONE}

\textcolor{preprocessor}{#include <websocketpp/config/asio\_no\_tls.hpp>}
\textcolor{preprocessor}{#include <websocketpp/server.hpp>}

\textcolor{preprocessor}{#include <functional>}

\textcolor{keyword}{typedef} \mbox{\hyperlink{classwebsocketpp_1_1server}{websocketpp::server<websocketpp::config::asio>}} 
      \mbox{\hyperlink{classwebsocketpp_1_1server}{server}};

\textcolor{keyword}{class }\mbox{\hyperlink{classutility__server}{utility\_server}} \{
\textcolor{keyword}{public}:
    \mbox{\hyperlink{classutility__server}{utility\_server}}() \{
         \textcolor{comment}{// Set logging settings}
        m\_endpoint.\mbox{\hyperlink{classwebsocketpp_1_1endpoint_a8292bcdca9344b57af1b0964ff7fc778}{set\_error\_channels}}(
      \mbox{\hyperlink{structwebsocketpp_1_1log_1_1elevel_a9b31ff708c221d314f9f4eb3ff2b1ad7}{websocketpp::log::elevel::all}});
        m\_endpoint.\mbox{\hyperlink{classwebsocketpp_1_1endpoint_a5d7da609ebd2f797e5e67b6d050ebc59}{set\_access\_channels}}(
      \mbox{\hyperlink{structwebsocketpp_1_1log_1_1alevel_a853aa0b8976e53f3181af3bc398d493e}{websocketpp::log::alevel::all}} ^ 
      \mbox{\hyperlink{structwebsocketpp_1_1log_1_1alevel_aa38cfdf7a82f33cac319438462707e90}{websocketpp::log::alevel::frame\_payload}});

        \textcolor{comment}{// Initialize Asio}
        m\_endpoint.init\_asio();
    \}

    \textcolor{keywordtype}{void} run() \{
        \textcolor{comment}{// Listen on port 9002}
        m\_endpoint.listen(9002);

        \textcolor{comment}{// Queues a connection accept operation}
        m\_endpoint.\mbox{\hyperlink{classwebsocketpp_1_1server_a0204a7d444144f7ea5b8bbcf14689fc1}{start\_accept}}();

        \textcolor{comment}{// Start the Asio io\_service run loop}
        m\_endpoint.run();
    \}
\textcolor{keyword}{private}:
    \mbox{\hyperlink{classwebsocketpp_1_1server}{server}} m\_endpoint;
\};

\textcolor{keywordtype}{int} main() \{
    \mbox{\hyperlink{classutility__server}{utility\_server}} s;
    s.run();
    \textcolor{keywordflow}{return} 0;
\}
\end{DoxyCode}


\subsubsection*{Step 2}

{\itshape Set up a message handler to echo all replies back to the original user}

\paragraph*{Setting a message handler}

\begin{quote}
\subparagraph*{Terminology\+: Registering handlers}

Web\+Socket++ provides a number of execution points where you can register to have a handler run. Which of these points are available to your endpoint will depend on its config. T\+LS handlers will not exist on non-\/\+T\+LS endpoints for example. \mbox{\hyperlink{struct_a}{A}} complete list of handlers can be found at \href{http://www.zaphoyd.com/websocketpp/manual/reference/handler-list}{\tt http\+://www.\+zaphoyd.\+com/websocketpp/manual/reference/handler-\/list}.

Handlers can be registered at the endpoint level and at the connection level. Endpoint handlers are copied into new connections as they are created. Changing an endpoint handler will affect only future connections. Handlers registered at the connection level will be bound to that specific connection only.

The signature of handler binding methods is the same for endpoints and connections. The format is\+: {\ttfamily set\+\_\+$\ast$\+\_\+handler(...)}. Where $\ast$ is the name of the handler. For example, {\ttfamily set\+\_\+open\+\_\+handler(...)} will set the handler to be called when a new connection is open. {\ttfamily set\+\_\+fail\+\_\+handler(...)} will set the handler to be called when a connection fails to connect.

All handlers take one argument, a callable type that can be converted to a {\ttfamily std\+::function} with the correct count and type of arguments. You can pass free functions, functors, and Lambdas with matching argument lists as handlers. In addition, you can use {\ttfamily std\+::bind} (or {\ttfamily boost\+::bind}) to register functions with non-\/matching argument lists. This is useful for passing additional parameters not present in the handler signature or member functions that need to carry a \textquotesingle{}this\textquotesingle{} pointer.

The function signature of each handler can be looked up in the list above in the manual. In general, all handlers include the {\ttfamily connection\+\_\+hdl} identifying which connection this even is associated with as the first parameter. Some handlers (such as the message handler) include additional parameters. Most handlers have a void return value but some ({\ttfamily validate}, {\ttfamily ping}, {\ttfamily tls\+\_\+init}) do not. The specific meanings of the return values are documented in the handler list linked above. \end{quote}


\subsubsection*{Step 3}

{\itshape error handling}

\subsubsection*{Step 4}

{\itshape Set up open and close handlers and a connection data structure}

\subsubsection*{Step 5}

{\itshape Change the message handler for connections based on U\+RI and add a validate handler to reject invalid U\+R\+Is}

\subsubsection*{Step 6}

{\itshape Add some Admin commands (report total clients, cleanly shut down server)}

\subsubsection*{Step 7}

{\itshape Add some Broadcast commands}

\subsubsection*{Step 8}

{\itshape Add T\+LS} 